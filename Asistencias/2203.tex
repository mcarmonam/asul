\documentclass[11pt, a4paper]{report}

\usepackage[utf8]{inputenc}
\usepackage{fancyvrb}
\usepackage{enumitem}
\usepackage{hyperref}
\usepackage{multirow}
\usepackage{graphicx}

\begin{document}
\title{22/Marzo/2019}
\author{
  Carmona Mendoza Mat\'in\\
  \texttt{313075977}
}
\date{}
\maketitle

En la clase de hoy se habló de iptables la cual es una parte del kernel de
Linux, está integrado en el, desde la versión 3 sino recuerdo mal. Es un filtro
de paquetes de red, podemos decir que es un firewall. \\

Iptables utiliza las tablas de enrutamiento. Estas tablas es la forma en la que
iptables se organiza para funcionar. Estas tablas albergan reglas que permiten
aplicar distintos tratamientos a los paquetes de red que llegen a nuestro
equipo. \\

Las tablas FILTER y NAT son las tablas más importantes para los
administradores. FILTER es el filtro y NAT network address translation
(muchos diccionarios, wikipedias, etc traducen translation como traducción pero
más bien, mi parecer es que la correcta traducción sería traslación Efecto de
trasladar o trasladarse de lugar, ya que de lo que se trata, es de trasladar
los paquetes de red de una dirección IP a otra. Seguro que alguien coincide
conmigo sobre esta mala traducción de anglicismo) es la tabla que se encarga de
tratar los paquetes cuando estos requieren un tratamiento más específico. \\


\newpage

\textbf{Bibliografía} \\

\begin{itemize}

\item \url{https://elbinario.net/2019/03/18/iptables-para-torpes/}


\end{itemize}


\end{document}
