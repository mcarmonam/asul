\documentclass[11pt, a4paper]{report}

\usepackage[utf8]{inputenc}
\usepackage{fancyvrb}
\usepackage{enumitem}
\usepackage{hyperref}
\usepackage{multirow}
\usepackage{graphicx}

\begin{document}
\title{Jus For Fun}
\author{
  Carmona Mendoza Mat\'in\\
  \texttt{313075977}
}
\date{}
\maketitle

Esta es la historia de Linus Benedict Torvalds, el creador de Linux OS en lo
que parecen ser sus propias palabras (el 90\% del libro está escrito como si el
propio Linus lo estuviera narrando). Lo que más me interesó y me mantuvo
leyendo el libro fue saber cómo Linus continuó su autoaprendizaje de la
informática. Comenzó escribiendo juegos y programas de juguetes en lenguaje
ensamblador, luego se enseñó a sí mismo y siguió haciendo proyectos para
dominar sus habilidades. Uno de los proyectos fue un emulador de terminal que
escribió en Minix OS. Continuó agregándole características y gradualmente
terminó haciendo un sistema operativo en cuestión de meses. Comenzó con Minix
después de leer el libro de Andy Tanenbaum sobre Diseño e implementación de
sistemas operativos, 3ª edición, Prentice Hall Software Series, que fue el
libro que Linus dice que cambió su vida. \\

Una buena lectura para todos los programadores que gustan de Linux. Como dice
Bertrand Russel: se puede obtener mucho placer de un conocimiento inútil, por
lo que incluso si este libro no habla de detalles técnicos del sistema
operativo, pero saber cómo llegó a ser lo que es, es interesante y es un placer
darlo todo. \\

En verdad lo recomiendo.


\newpage

\textbf{Bibliografía} \\

\begin{itemize}

\item \url{https://github.com/limkokhole/just-for-fun-linus-torvalds/blob/master/just-for-fun.pdf?fbclid=IwAR0Ce44o2PYIByfxzp4Ezdu5Zdga7OHnillLy5gTIi4_WO8KAt9acwsN37A}


\end{itemize}


\end{document}
