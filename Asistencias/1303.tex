\documentclass[11pt, a4paper]{report}

\usepackage[utf8]{inputenc}
\usepackage{fancyvrb}
\usepackage{enumitem}
\usepackage{hyperref}
\usepackage{multirow}
\usepackage{graphicx}

\begin{document}
\title{Clase 32: 13/Marzo/2019}
\author{
  Carmona Mendoza Mat\'in\\
  \texttt{313075977}
}
\date{}
\maketitle

Hoy en clase seguimos trabajando con las máquinas virtuales. Después de
resolver algunos los problemas con el inicio de las máquinas virtuales realizamos la copia de nuestra llave publica al directorio \/.ssh de la cuenta user de
cada máquina virtual, para no tener que hacer esto dirección por dirección nos
ayudamos de un ciclo for dentro de la línea de comandos. \\

También platicamos acerca de que ya que un servidor DHCP esta asignando
las direcciones IP de manera dinámica en el segmento de red sobre el que están
las máquinas virtuales, ¿como podríamos hacer? para mantener una lista
actualizada con únicamente las direcciones IP de demás máquinas virtuales de
manera que siempre que necesitáramos conectarnos con una lo pudiéramos hacer sin
tener que revisar que dirección IP tiene. \\

Como información adicional busqué información acerca del protocolo ARP y
su funcionamiento y el ataque arp spoof ing. \\

ARP: Es un protocolo de resolución de direcciones. Permite encontrar la
dirección hardware equivalente a una determinada dirección IP, actuando como
traductor e intermediario. Cuando una máquina desea ponerse en contacto con
otra y no se conoce su dirección IP entonces envía una petición ARP por
broadcast (o sea a todas las maquinas). El protocolo establece que solo
contestara a la petición, si esta lleva su dirección IP. Por lo tanto solo
contestará la máquina
que corresponde a la dirección IP buscada, con un mensaje que incluya la
dirección física. El software de comunicaciones debe mantener una caché con los
pares IP-dirección física. De este modo la siguiente vez que hay que hacer una
transmisión a es dirección IP, ya conoceremos la dirección física. \\

Ataque ARP Spoofing: Un ARP Spoofing es una especie de ataque en el que
un atacante envía mensajes falsificados ARP (Address Resolution Protocol) a
una LAN. \\

Como resultado, el atacante vincula su dirección MAC con la dirección IP
de un equipo legítimo (o servidor) en la red. \\

Si el atacante logró vincular su dirección MAC a una dirección IP auténtica,
va a empezar a recibir cualquier dato que se puede acceder mediante la dirección
IP. \\

ARP Spoofing permite a los atacantes maliciosos interceptar, modificar o
incluso retener datos que están en tránsito. Los ataques de suplantación ARP
ocurren en redes de área local que utilizan protocolo de resolución de
direcciones (ARP).

\newpage

\textbf{Bibliografía} \\

\begin{itemize}

\item \url{https://medium.com/@marvin.soto/qu\%C3\%A9-es-el-envenenamiento-arp-
  o-ataque-arp-spoofing-y-c\%C3\%B3mo-funciona-7f1e174850f2}
\item \url{https://www.ecured.cu/Protocolo de resoluci\%C3\%B3n de direcciones}
\end{itemize}


\end{document}
