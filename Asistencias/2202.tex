\documentclass[11pt, a4paper]{report}

\usepackage[utf8]{inputenc}
\usepackage{fancyvrb}
\usepackage{enumitem}
\usepackage{hyperref}

\begin{document}
\title{Clase 19: 22/Febrero/2019}
\author{
  Carmona Mendoza Mat\'in\\
  \texttt{313075977}
}
\date{}
\maketitle

En la clase de hoy continuamos hablando de la carpeta dev que como recordamos
es el directorio que contiene los ficheros de dispositivo. \\

Los dispositivos están divididos en dos tipos: los dispositivos de carácter y
los dispositivos de bloque. \\

Los dispositivos de carácter son accedidos secuencialmente, un carácter cada
vez. Algunos ejemplos de dispositivos de carácter son el ratón, el teclado,
un terminal de texto, una cinta magnética, nulo, etc. \\

Los dispositivos de bloque se caracterizan por ser de acceso aleatorio, la
unidad mínima de lectura-escritura no es un carácter, sino un bloque (1KB).
Algunos ejemplos de dispositivos de carácter son los discos duros, los
disquetes, los CDROMS, etc. \\

La diferencia es que los dispositivos de bloque tienen un búfer para las
peticiones, por lo tanto pueden escoger en qué orden las van a responder. Esto
es importante en el caso de los dispositivos de almacenamiento, donde es más
rápido leer o escribir sectores que están cerca entre sí, que aquellos que
están más desperdigados. Otra diferencia es que los dispositivos de bloque
sólo pueden aceptar bloques de entrada y de salida (cuyo tamaño puede variar
según el dispositivo), en cambio los dispositivos de carácter pueden usar
muchos o unos pocos bytes como ellos quieran. La mayoría de los dispositivos
del mundo son de carácter, porque no necesitan este tipo de buffering, y no
operan con un tamaño de bloque fijo. Se puede saber cuándo un fichero de
dispositivo es para un dispositivo de carácter o de bloque mirando el primer
carácter de la salida de ls -l. Si es `b' entonces es un dispositivo de
bloque, y si es `c' es un dispositivo de carácter. \\

Hablamos tambien de la carpeta etc teniendo esta los ficheros de configuración,
scripts de arranque, etc. algunos de los archivos que podemos encontrar aquí
son los siguientes: \\

\begin{itemize}
\item /etc/passwd : La base de datos de los usuarios, que incluye campos como
  el nombre de usuario, nombre real, directorio home, password encriptada y
  otra información acerca de cada usuario. El formato de este archivo se
  encuentra documentado en la página de manual del comando passwd. Sin
  embargo, hoy día es muy común encontrar las contraseñas encriptadas en
  /etc/shadow. Esto significa que en tal caso, los datos de los usuarios
  excepto la password encriptada se encontrarían almacenados en passwd.
\item /etc/fdprm : Tabla de parámetros para los discos flexibles. Describe
  cómo son los diferentes formatos de estos discos. Este archivo es utilizado
  por el programa setfdprm. Se puede encontrar información adicional en la
  página de manual de setfdprm.
\item /etc/magic : El archivo de configuración para el programa file. Contiene
  las descripciones de varios formatos de archivos que utiliza file para
  determinar el tipo de archivo.
\item /etc/mtab : Contiene un listado de los sistemas de archivos actualmente
  montados. Se establece Inicialmente por los scripts del arranque del sistema, y se actualiza automáticamente por el comando mount. Se utiliza cuando se
  necesita un listado de los sistemas de archivos que estén actualmente
  montados (por ejemplo por el comando df). 
\end{itemize}

Hablamos del comando fdisk, siendo una  herramienta para crear, eliminar,
redimensionar, cambiar o copiar y mover particiones usando el sencillo menú
que ofrece. El límite que existe en esta herramienta está en 4 particiones
primarias como máximo por disco , y un número de particiones extendidas o
lógicas que será variable en función del tamaño de nuestro disco duro. \\

Algunas banderas para este comando son: \\

\begin{itemize}
\item fdisk –l : Para listar todas las particiones existentes en nuestro
  sistema.
\item fdisk -l /dev/sdb/ : Para ver todas las particiones de un único disco.
\item fdisk –s /dev/sdb2 :  Comprobar el tamaño de una partición.    
\end{itemize}

Vimos también controladores de sistemas de archivos. Mencionando: \\

\begin{itemize}
\item msdos : El controlador del sistema de archivos msdos no proporciona una
  semántica de archivos Unix adicional y no es compatible con un nombre de
  archivo largo. Si se monta un sistema de archivos de disco FAT usando este
  controlador, solo serán visibles los nombres de archivo 8.3, no se podrá
  acceder a los nombres de archivo largos, ni se mantendrán estructuras de
  datos de nombres de archivo de ningún tipo en el volumen del disco. El
  controlador del sistema de archivos vfat proporciona soporte de nombre de
  archivo largo utilizando las mismas estructuras de datos de disco que
  Microsoft Windows usa para soporte de nombre de archivo largo VFAT en
  volúmenes de formato FAT, pero no admite ninguna semántica de archivo Unix
  adicional. El controlador del sistema de archivos umsdos proporciona soporte
  de nombre de archivo largo, y semántica de archivos Unix extra. Sin embargo,
  lo hace utilizando estructuras de datos en disco que no son reconocidas por
  ningún controlador de sistema de archivos para ningún sistema operativo que
  no sea Linux
\item umsdos : La ventaja clave para umsdos de los tres es que proporciona una
  semántica completa de archivos Unix. Por lo tanto, se puede usar en
  situaciones en las que es deseable instalar Linux y ejecutarlo desde un
  volumen de disco FAT, que requiere dicha semántica para estar disponible.
  Sin embargo, Linux instalado y ejecutándose desde dicho volumen de disco es
  más lento que Linux instalado y ejecutándose desde un volumen de disco
  formateado, por ejemplo, con el formato del sistema de archivos ext2 
\item vfat : aunque carece de la semántica completa de archivos Unix y no tiene
  la capacidad de tener Linux instalado y ejecutándose desde un volumen de
  disco FAT, no tiene las desventajas mencionadas anteriormente de umsdos
  cuando se trata simplemente de compartir datos en un volumen de disco FAT
  entre Linux y otros sistemas operativos. sistemas tales como Windows. Sus
  estructuras de datos son las mismas que las utilizadas por Windows para los
  nombres de archivo largos de VFAT, y no requiere ejecutar una utilidad de
  sincronización para evitar que las estructuras de datos de Windows y Linux
  se separen. Por esta razón, es el controlador de sistema de archivos FAT más
  apropiado para usar en la mayoría de las situaciones.
\end{itemize}



\newpage

\textbf{Bibliografía} \\

\begin{itemize}

\item  \url{https://reynaldo-entrada-salida.es.tl/Dispositivos-de-Bloqueo-y-Car%E1cter.htm}

\item \url{http://www.tldp.org/pub/Linux/docs/ldp-archived/system-admin-guide/translations/es/html/ch04s03.html}

\item \url{https://openwebinars.net/blog/9-comandos-basicos-fdisk-para-gestionar-el-disco-duro/}

\end{itemize}


\end{document}
