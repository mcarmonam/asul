\documentclass[11pt, a4paper]{report}

\usepackage[utf8]{inputenc}
\usepackage{fancyvrb}
\usepackage{enumitem}
\usepackage{hyperref}
\usepackage{multirow}
\usepackage{graphicx}

\begin{document}
\title{Clase 48: 08/Abril/2019}
\author{
  Carmona Mendoza Mat\'in\\
  \texttt{313075977}
}
\date{}
\maketitle
El profesor comenzó la clase dejándonos una "tarea moral", por así decirlo, la
cual consistía en responder dudas a los compañeros de nuevo ingreso sobre qué
distribución de Linux es más conveniente (para ellos) usar como principiante.
Algunas de las razones que como grupo fueron las siguientes: \\

\begin{itemize}
\item Caracteristicas de la arquitectura del computador: por ejemplo la
  capacidad de la memoria RAM, capacidad del disco duro, tamaño de la
  instrucción (32 bits o 64 bits), el precesador, etc.  
\item Conocimiento sobre UNIX (comandos, instalación de paquetes, etc).
\item Gustos personales (Escritorio, ventanas, etc.).
\item Amigabilidad de la distribución.
\item Propósitos para los cuales usarás la distribución.
\end{itemize}

\section*{Distribuciones Estables}

\begin{itemize}
\item Red Hat Enterprise Linux: Distribución comercial de Linux desarrollada
  por Red Hat. Ofrece una estabilidad y flexibilidad punteras, lo que la coloca
  como una de las más recomendadas para empresas y servidores.
\item Debian: Muy estable y 100\% libre, Debian destaca por su sistema de
  paquetería .deb y su gestión de paquetes APT. Es una de las distribuciones
  más importantes de GNU/Linux, ya que en ella se basan gigantes como Ubuntu.
\item openSUSE: Es una de las alternativas más potente contra la familia de
  distribuciones basadas en Debian. Está disponible con los entornos de
  escritorio KDE y Gnome, y cuenta como una de sus mejores armas con la robusta
  herramienta de instalación y configuración YaST y el configurador gráfico SaX.
\item Fedora: Distribución gratuita creada y mantenida por la empresa Red Hat
  que utiliza el sistema de paquetería RPM (Red Hat Package Manager). Tiene
  tres versiones diferentes para escritorio, servidores y sistemas en la nube,
  y destaca por su seguridad gracias al sistema SELinux ("Security-Enhanced
  Linux").
\item CentOS: Nació como un derivado gratuito de la distribución comercial Red
  Hat Enterprise Linux (RHEL) destinada al uso empresarial. Recientemente unió
  las fuerzas con el propio Red Hat, y sigue siendo una apuesta segura para los
  que busquen un código de gran calidad.
\item Arch Linux: Una distribución modular en la que empiezas desde cero y
  tienes que ir añadiéndole los componentes que quieras. No es muy apta para
  principiante, y utiliza pacman, su propio gestor de paquetes. Se trata de una
  Rolling Release, lo que quiere decir que todos sus componentes van
  actualizándose sin necesidad de instalar versiones nuevas del sistema
  operativo.
\item Manjaro: Una prometedora distribución que promete llevar todo el
  potencial de Arch Linux al usuario menos experimentado. Para eso, ofrece un
  sistema operativo ya montado y basado en Arch, con un instalador sencillo
  como el que podemos encontrar en otras distribuciones como Ubuntu. Tiene
  versiones oficiales con los entornos de escritorio XFCE y KDE.
\end{itemize}

\section*{Distribuciones para Principiantes}

\begin{itemize}
\item Ubuntu: Una de las distribuciones más utilizadas gracias a su gran
  facilidad de uso. Basada en Debian, es amada y odiada por partes iguales por
  su exclusivo entorno de escritorio Unity, con el que persigue convertirse en
  una distribución versátil que pueda utilizarse tanto en ordenadores como
  móviles y tabletas.
\item Linux Mint: Basado en Ubuntu, es uno de los más recomendados para todos
  aquellos que tocan Linux por primera vez. Su entorno de escritorio, Cinnamon,
  tiene muchas similitudes con el de Windows, y es también uno de los más
  personalizables.
\item Elementary OS: De todas las distribuciones basadas en Ubuntu, esta una de
  las que más personalidad tiene gracias a su cuidadísimo aspecto, que imita el
  del sistema operativo OS X de Apple. Increíblemente rápida y fácil de usar,
  le ofrece al usuario todo lo que pueda necesitar desde el primer momento,
  incluyendo una colección de aplicaciones propias diseñadas para integrarse a
  la perfección con su estilo visual.
\item Zorin OS: Distribución también basada en Ubuntu que nació con la
  intención de ayudar a que el usuario diera el salto a Linux ofreciéndole una
  interfaz lo más similar posible a Windows. Tiene varias versiones, algunas
  gratuitas como Core (versión básica), Lite (para PCs poco potentes) y
  Educational (incluye aplicaciones educativas), y otras cuantas de pago muy al
  estilo de las versiones de Windows.
\item Peppermint OS: Distribución rápida y ligera basada en Ubuntu con entorno
  de escritorio LXDE. Utiliza la tecnología Prism de Mozilla para integrarse
  con las aplicaciones basadas en la nube, utilizando las webapps como si
  fueran nativas. Se presenta como una alternativa a otros sistemas basados en
  la nube como Chrome OS.
\end{itemize}

\section*{Distribuciones con buena Privacidad}

\begin{itemize}
\item Tails: Promocionada por el propio Edward Snowden y basada en Debian, es
  una distribución lista para ser ejecutada desde un USB o DVD. Tails se
  conecta a TOR tan pronto termina el proceso de inicio del sistema operativo,
  y toda conexión a Internet se realiza a través de esta red.
\item Kali Linux: Distribución basada en Debian con una inmensa colección de
  herramientas para proteger nuestros equipos. Utiliza un kernel personalizado
  con parches de seguridad y tiene soporte para la arquitectura ARM.
\item BlackArch Linux: Una distribución orientada a la seguridad informática
  que en un principio nació como expansión de Arch Linux, pero que ha seguido
  su propio camino. Nos da acceso a una impresionante cantidad de herramientas
  de hacking entre las que destaca Sploitctl, un script que permite instalar,
  actualizar y buscar sploits.
\item Arch Assault: Se trata de una nueva distribución, también basada en Arch
  Linux y muy parecida a la anterior, también dirigida a hackers y amantes de
  la seguridad. Minimalista, con gestor de ventanas Openbox acompañado por el
  panel Tint2, a pesar de estar aun verde ya ofrece soporte para arquitecturas
  ARM.
\end{itemize}

\section*{Distribuciones para Equipos no tan Potentes}

\begin{itemize}
\item Puppy Linux: Una minúscula distribución que puede llevarse en un USB o
  CD, pero sorprendentemente rápida al cargarse enteramente en la memoria RAM
  del ordenador. Se carga en 30 o 40 segundos y ocupa sólo 100 MB.
\item Lubuntu: Se trata de una versión de Ubuntu mucho más ligera y asequible
  para equipos poco potentes al utilizar el sistema de escritorio LXDE y el
  gestor de ventanas Openbox. También incluye software personalizado bastante
  más ligero, por lo que sólo nos pide 128MB de RAM y un Pentium II o Celeron
  de 1999 para funciona.
\item Damn Small Linux: Distribución especialmente diseñada para los equipos
  más antiguos, como los Pentium de primera generación o incluso los i486. Como
  entorno gráfico y gestor de ventanas nos propone JWM, su iso ocupa apenas 50
  MB y sólo nos pide como mínimo un Intel 486DX y 16 MB de memoria RAM.
\item SliTaz: Otro peso pluma aunque con un software ligeramente más moderno
  que el del anterior. Utiliza el entorno Openbox y sólo necesita un Pentium III con 256MB de RAM y 100 MB libres en el disco duro para funcionar.
\item LXLE: Basada en Lubuntu, esta distribución promete ser aun más ligera
  gracias a un mejor procesado de inicio y el entorno de escritorio LXDE.
  Ofrece varios perfiles que amoldarán la distro para que se parezca a Windows
  XP, Vista, y 7 Starter/Basic.
\item Bodhi Linux: Aunque actualmente su desarrollo está paralizado después de
  que su creador abandonase el barco, aun podemos utilizar las últimas
  versiones de esta distribución para nuestros equipos antiguos. Utiliza un
  entorno de escritorio Enlightenment y sólo pide como mínimo un equipo con
  procesador de 300 MHz, 128 MB de RAM y 2,5 GB de espacio libre en el disco
  duro.
\item Q4OS: Y si el anterior era un proyecto que llegaba a su fin, Q4OS es uno
  que está comenzando. Se trata de una distro basada en Debian. Su entorno de
  escritorio deriva de la una versión 3.x de KDE llamada Trinity DE e imita el
  aspecto de Windows XP. Puede usarse en equipos con Pentium de 300MHz, 128 MB
  de RAMy 3 GB de disco duro.
\end{itemize}

\section*{Distribuciones para distintos Hobbies}

\begin{itemize}
\item Distro Astro: Esta distribución está basada en Ubuntu 14.04 LTS y utiliza
  el entorno de escritorio MATE, aunque lo más importante es su completa
  colección de aplicaciones dirigidas a los amantes de la astronomía.
\item SteamOS: Aun en fase beta, esta es la distribución basada en Debian
  desarrollada por Valve, y que está más dirigida a ser una especie de media
  center para videojuegos integrando el modo big picture que un sistema de
  sobremesa.
\item ArtistX: Distribución de Linux enfocada en la producción multimedia.
  Basada en Ubuntu, aunque diseñada para ser utilizada en formato LiveDVD y
  USB, se puede instalar en cualquier equipo. Usa el entorno KDE y ofrece una
  colección de programas de código abierto para edición de vídeo y creación de
  gráficos 2D y 3D.
\item Ubuntu Studio: Basada en Ubuntu y orientada a la edición multimedia
  profesional de audio, video y gráficos. Utiliza el entorno de escritorio Xfce
  y no lleva preinstalado ningún tipo de software ofimático, sólo el diseñado
  para la edición multimedia.
\item Scientific Linux: Se trata de un clon a nivel binario de la distribución
  Red Hat Enterprise Linux, y está desarrollada y mantenida por los
  laboratorios de Física CERN y Fermilab con el objetivo de tener un sistema
  operativo específico para la computación científica.
\item CEELD: Distro basada en OpenSUSE que usa el entorno KDE y está
  especialmente dirigida a los ingenieros electrónicos y a estudiantes de esta
  carrera, al permitirles diseñar o simular circuitos electrónicos.
\item Edubuntu: Otro derivado de Ubuntu, pero en esta ocasión especialmente
  dirigido a escuelas y profesores. Ofrece una gran colección de software y
  herramientas educativas, por lo que también es una buena opción para instalar
  en los ordenadores de los más jóvenes de la casa.
\item Openelec: Pequeña distribución Linux creada desde cero para convertir un
  ordenador en un centro multimedia basado en Kodi, lo que antes se conocía
  como XBMC. Hace lo que promete y necesita sólo 90-125 MB de almacenamiento
  interno. A parte de su versión oficial, tiene otras dos buids para Raspberry
  Pi y Apple TV.
\end{itemize}

\newpage

\textbf{Bibliografía} \\

\begin{itemize}

\item \url{https://www.genbeta.com/linux/31-distribuciones-de-linux-para-elegir-bien-la-que-mas-necesitas}
\end{itemize}


\end{document}
