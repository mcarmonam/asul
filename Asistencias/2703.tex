\documentclass[11pt, a4paper]{report}

\usepackage[utf8]{inputenc}
\usepackage{fancyvrb}
\usepackage{enumitem}
\usepackage{hyperref}
\usepackage{multirow}
\usepackage{graphicx}

\begin{document}
\title{27/Marzo/2019}
\author{
  Carmona Mendoza Mat\'in\\
  \texttt{313075977}
}
\date{}
\maketitle

En la clase de hoy con el profesor continuamos con la creación de una máquina
virtual con Windows 10, por motivos de tiempo fue mejor copiar una máquina virtual ya con la instalación realizada y no hacerla nosotros mismos. \\

Lo siguiente que hicimos fue compartir un directorio desde nuestra máquina con
Debian mediante samba, para esto modificamos el archivo $/etc/samba/smb.conf$
para indicarle el segmento y la interfaz de red, después indicamos el nombre
con el que se comparte el directorio así como ciertas especificaciones para
este como lo son permisos de solo lectura, permisos para escribir, etc. \\

\textbf{Samba} \\

La creación de Samba es la idea de Andrew Tridgell. Es un proyecto que nació en
1991 cuando creó un programa servidor de archivos para su red local, la cual
soportaba un protocolo reconocido como DEC de Digital Pathworks. Aunque en ese
momento el no lo supo, dicho protocolo se convertiría en SMB luego. \\

Samba, básicamente, es una suite de aplicaciones Unix que implementa el
protocolo SMB (Server Message Block). Este protocolo es empleado para
operaciones cliente-servidor en una red. Entonces, mediante el uso de este
protocolo Samba le permite a Unix establecen comunicación con productos
Microsoft Windows a través del protocolo. De esta manera, una maquina Unix con
Samba puede ingresar a la red Microsoft, mostrándose como Servidor y brindar
los siguientes servicios:

\begin{itemize}
\item Compartir diversos sistemas de archivos.
\item Compartir impresoras, con instalación en el servidor como en los clientes.
\item Proveer un visualizados de clientes en red, lo que facilitara la colaboración con nuestros usuarios.
\item Permite realizar verificación de clientes a través de un login contra un dominio Windows.
\item Proporcionar o asistir con un servidor de resolución de nombres WINS.
\end{itemize}

\newpage

\textbf{Bibliografía} \\

\begin{itemize}

\item \url{https://www.profesionalreview.com/2017/03/25/servidor-samba-conceptos-y-configuracion-rapida/}


\end{itemize}


\end{document}
