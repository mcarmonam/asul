\documentclass[11pt, a4paper]{report}

\usepackage[utf8]{inputenc}
\usepackage{fancyvrb}
\usepackage{enumitem}
\usepackage{hyperref}
\usepackage{multirow}
\usepackage{graphicx}

\begin{document}
\title{19/Marzo/2019}
\author{
  Carmona Mendoza Mat\'in\\
  \texttt{313075977}
}
\date{}
\maketitle

En la clase de hoy con el profesor hablamos sobre el kernel de windows. El
kernel es un componente fundamental de cualquier sistema operativo. Es el
encargado de que el software y el hardware de cualquier ordenador puedan
trabajar juntos en un mismo sistema, para lo cual administra la memoria de los
programas y procesos ejecutados, el tiempo de procesador que utilizan los
programas, o se encarga de permitir el acceso y el correcto funcionamiento de
periféricos y otros elementos físicos del equipo. \\

En la década de los noventa Microsoft estaba basando sus sistemas operativos en
los kernel Windows 9x, donde el código básico tenía muchas similitudes con
MS-DOS. De hecho necesitaba recurrir a él para poder operar. Paralelamente,
Microsoft también estaba desarrollando otra versión de su sistema dirigido a
los servidores llamada Windows NT, que nació el 26 de julio de 1993. \\

Ambas versiones de Windows fueron desarrollándose por separado. Windows NT era
más bien una jugada a largo plazo, una tecnología que ir desarrollando para los
Windows del mañana, y en el año 2000 dieron un nuevo paso en esa dirección. A
la versión 5.0 de NT la llamaron Windows 2000, y se convirtió en un interesante
participante en el sector empresarial. \\

Tras ver la buena acogida que tuvo, Microsoft decidió llevar NT al resto de
usuarios para que ambas ramificaciones convergieran. Lo hicieron en octubre del
2001 con la versión 5.1 de Windows NT, que llegó al mercado con el nombre de
Windows XP. Por lo tanto, esta versión marcó un antes y un después no sólo por
su gran impacto en el mercado, sino porque era el principio de la aventura del
Kernel Windows NT en el mundo de los usuarios de a pie. \\

Desde ese día, todas las versiones de Windows han estado basadas en este Kernel
con más de 20 añós de edad. La versión 5.1.2600 fue Windows XP, la 6.0.6002 fue
Windows Vista, y la 6.1.7601 Windows 7. Antes hubo otros Windows Server 2008 y
2003, y después llegaron las versiones de NT 6.2.9200 llamada Windows 8, la
6.3.9600 o Windows 8, y finalmente la NT 10.0, también conocida como Windows
10. \\

\section*{Diferencias entre el kernel de Linux y de Windows} 

La principal diferencia entre el Kernel de los sistemas operativos Windows y el
de Linux está en su filosofía. El desarrollado por el equipo de Linus Torvalds
es de código abierto y cualquiera puede cogerlo y modificarlo, algo que le
sirve para estar presente en múltiples sistemas operativos o distros GNU/Linux.
El de Microsoft en cambio es bastante más cerrado, y está hecho por y para el
sistema operativo Windows. \\

En esencia, en Linux cogieron los principios de modularidad de Unix y
decidieron abrir el código y las discusiones técnicas. Gracias a ello, Linux ha
creado una comunidad meritocrática de desarrolladores, una en la que todos
pueden colaborar y en la que cada cambio que se sugiere se debate con dureza
para desechar las peores ideas y quedarse con las mejores. También se halaga a
quienes consiguen mejorar las funcionalidades más veteranas. \\

Mientras, en Windows la cosa no funciona así. Tal y como explicaba en un
artículo un antiguo colaborador del Kernel, los responsables del Kernel no ven
con buenos ojos que se hagan propuestas que se desvíen del plan de trabajo, y
asegura que hay pocos incentivos para mejorar las funcionalidades existentes
que no sean prioritarias. \\

Esto hace, a ojos de ese antiguo desarrollador, que al dársele mayor
importancia a cumplir planes que a aceptar cambios que mejoren la calidad del
producto, o al no tener tantos programadores sin experiencia, el Kernel de
Windows NT siempre esté un paso por detrás en estabilidad y funcionalidades. \\


\newpage

\textbf{Bibliografía} \\

\begin{itemize}

\item \url{https://www.genbeta.com/a-fondo/como-es-el-kernel-de-windows-y-cuales-son-sus-diferencias-con-el-de-linux}

\end{itemize}


\end{document}
