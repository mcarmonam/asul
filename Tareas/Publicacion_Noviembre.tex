\documentclass[11pt, a4paper]{report}

\usepackage[utf8]{inputenc}
\usepackage{fancyvrb}
\usepackage{enumitem}
\usepackage{hyperref}
\usepackage{multirow}
\usepackage{graphicx}

\begin{document}
\title{Publicaciones del mes de Noviembre}
\author{
  Carmona Mendoza Mat\'in\\
  \texttt{313075977}
}
\date{}
\maketitle

\section*{Publicaciones que seleccioné}
\begin{itemize}

\item Publicación 1.
  \begin{itemize}
  \item Ubuntu 19.04 Dubbed Disco Dingo
  \item Linux Journal.
  \item 01 Noviembre 2018.
  \item \url{https://bit.ly/31vHXH7}
  \end{itemize}
  
\item Publicación 2
  \begin{itemize}
  \item Que hacer después de instalar Ubuntu 18.10 Cosmic.
  \item SoloLinux.
  \item 03 Noviembre 2018. 
  \item \url{https://bit.ly/2KnzT5M}
  \end{itemize}

\item Publicación 3
  \begin{itemize}
  \item PlayStation Classic utiliza un emulador de código abierto.
  \item muyLinux.
  \item 12 Noviembre 2018.
  \item \url{https://bit.ly/2MOlDoL}
  \end{itemize}
\end{itemize}  


\section*{Descripción de la publicación}
Elegí la publicación 2, Que hacer después de instalar Ubuntu 18.10 Cosmic, ya
que en el momento que leí la nota yo tenía como distribución la que se
mencionaba en el escrito. La mayoria de las cosas que se mencionan ya las había
habilitado o instalado en mi equipo pero algo que me pareció bastante innovador
fue una aplicación de herramientas para poder para personalizar nuestro
escritorio, Gnome Tweaks. Tenía mucho tiempo buscando herramientas para poder
manipular mi escritorio y porfin lo encontré. \\

Considero que el articulo es muy bueno para los principiantes en el mundo de
Linux ya que te enseña cosas como cambiar a root desde la terminal, comandos
para actualizar nuestro Ubuntu y controladores, cómo instalar paquetes como Java
por ejemplo, como activar el firewall, como instalar aplicaciones.

\section*{Conceptos que tuve que investigar}
Llevo utilizando ya mucho tiempo Linux por lo que tengo un conocimiento sobre el
sistema operativo, es decir ya manejo muchos conceptos como herramientas de
Linux, menciono porque considero que el articulo que leí es de una dificultad
muy fácil por lo que entendí todo a la perfección y no tuve que buscar algun
concepto.

\section*{Personas a las que recomendaría el articulo}
Recomiendo el articulo para principiantes porque considero que no tiene gran
dificultad, ademas explica muy bien los temas para evitar confusión. Lo
recomiendo ampliamente para las personas que recien migraron a Linux y cuentan
con esta distribución.

\end{document}
