\documentclass[11pt, a4paper]{report}

\usepackage[utf8]{inputenc}
\usepackage{fancyvrb}
\usepackage{enumitem}
\usepackage{hyperref}
\usepackage{multirow}
\usepackage{graphicx}

\begin{document}
\title{Práctica 05: Compilar el Kernel}
\author{
  Carmona Mendoza Mat\'in\\
  \texttt{313075977}
}
\date{}
\maketitle

Para comenzar descargamos la última versión estable del kernel de la
página oficial kernel.org. \\

Para configurar el kernel utilizaremos la herramienta make menuconfig. \\

Para comenzar el proceso descomprimimos el código fuente del kernel que
descargamos anteriormente para poder copiar la configuración del kernel
actual a ese directorio. Todo esto se hace con los siguientes comandos: \\

  \begin{figure}[!ht] 
    \begin{center}  
      \includegraphics[width=1.1\textwidth]{1.png} 
      \caption{} 
    \end{center} 
  \end{figure} 

  Una vez hecho esto ejecutaremos el comando make menuconfig y se abrirá
  una interfaz que nos permitirá navegar las categorias de drivers. \\

    \begin{figure}[!ht] 
    \begin{center}  
      \includegraphics[width=1.0\textwidth]{2.png} 
      \caption{} 
    \end{center} 
  \end{figure} 

    Por lo tanto, es importante conocer bien nuestro equipo a fin de no
    deshabilitar drivers que serán necesarios. Comandos como lspci y
    dmesg pueden sernos útiles en este aspecto. \\

    La primera cosa que quité fue Touchscreen ya que tiene que ver con
    todo aquello de la pantalla tpactil y mi computadora no cuenta con esa
    función. \\

    \begin{figure}[!ht] 
      \begin{center}  
        \includegraphics[width=1.25\textwidth]{3.png} 
        \caption{} 
      \end{center} 
    \end{figure}
    
    Otra cosa que modifiqué fue Amateur radio support ya que es un driver
    para la configuración de protocolos para estaciones de radio. \\

    \begin{figure}[!ht] 
      \begin{center}  
        \includegraphics[width=1.25\textwidth]{4.png} 
        \caption{} 
      \end{center} 
    \end{figure}

    Quité también la configuración del disco duro ya que mi pc no cuenta
    con uno. \\

    \begin{figure}[!ht] 
      \begin{center}  
        \includegraphics[width=1.25\textwidth]{5.png} 
        \caption{} 
      \end{center} 
    \end{figure}

    Otra cosa que omití fue Macintosh device drivers el cual te ayuda
    para cofigurar dispositivos marca Mac. \\

    \begin{figure}[!ht] 
      \begin{center}  
        \includegraphics[width=1.25\textwidth]{6.png} 
        \caption{} 
      \end{center} 
    \end{figure}
    
    Desmarqué Laptop Hybrid Graphics ya que mi computador no tiene
    gráficos híbridos. \\
      
    \begin{figure}[!ht] 
      \begin{center}  
        \includegraphics[width=1.25\textwidth]{7.png} 
        \caption{} 
      \end{center} 
    \end{figure}

    Quité Nouveau (NVIDIA) cards porque mi pc no cuenta con una tarjeta
    gráfica Nvidia, tiene una intedrada. \\

    \begin{figure}[!ht] 
      \begin{center}  
        \includegraphics[width=1.25\textwidth]{8.png} 
        \caption{} 
      \end{center} 
    \end{figure}
   
    \newpage
    
    Omití Analog TV support ya que no tengo la tarjeta de TV/radio. \\

    \begin{figure}[!ht] 
      \begin{center}  
        \includegraphics[width=1.25\textwidth]{9.png} 
        \caption{} 
      \end{center} 
    \end{figure}
        
    Omití Digital TV support ya que no tengo la tarjeta de TV/radio. \\

    \begin{figure}[!ht] 
      \begin{center}  
        \includegraphics[width=1.25\textwidth]{10.png} 
        \caption{} 
      \end{center} 
    \end{figure}

    Omití AM/FM radio receivers ya que no tengo la tarjeta de TV/radio.\\
    
    \begin{figure}[!ht] 
      \begin{center}  
        \includegraphics[width=1.25\textwidth]{11.png} 
        \caption{} 
      \end{center} 
    \end{figure}    

    Para terminar compilamos la nueva configuración del kernel. \\

    \begin{figure}[!ht] 
      \begin{center}  
        \includegraphics[width=1.25\textwidth]{13.png} 
        \caption{} 
      \end{center} 
    \end{figure}    
    
\end{document}
