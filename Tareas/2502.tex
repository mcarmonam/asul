\documentclass[11pt, a4paper]{report}

\usepackage[utf8]{inputenc}
\usepackage{fancyvrb}
\usepackage{enumitem}
\usepackage{hyperref}

\begin{document}
\title{El debate de Tanenbaum-Torvalds}
\author{
  Carmona Mendoza Mat\'in\\
  \texttt{313075977}
}
\date{}
\maketitle

Andrew Stuart Andy Tanenbaum nacido el 16 de marzo de 1944 y conocido como ast
o papa Tanenbaum, es profesor de ciencias de la computación. \\

Tanenbaum es más conocido por ser el creador de Minix, una réplica gratuita del
sistema operativo UNIX con propósitos educativos, y por sus libros sobre
ciencias de la computación.\\

Por otra parte Linus Benedict Torvalds es un ingeniero de software finlandés;
conocido por iniciar y mantener el desarrollo del "kernel" Linux, basándose en
el sistema operativo libre Minix creado por Andrew S. Tanenbaum y en algunas
herramientas, los compiladores y un número de utilidades desarrollados por el
proyecto GNU.\\

En el año de 1992 estas dos celebridades tuvieron un gran debate, cabe
mencionar que otras celebridades tales como Peter Mc Donald y David S. Miller
también participaron.\\

El debate se produjo en un foro de discusión de Usenet, comp.os.Minix. En él,
Tanenbaum explica que los sistemas operativos de tipo micronúcleo (como Minix)
son mejores que los de tipo monolíticos (como Linux).\\

Todo empezó porque en 1991 Linus Torvalds, publicó un mensaje que decía algo
como:
"Hola a todos por ahí usando minix – Estoy haciendo un sistema operativo
(libre) (sólo un pasatiempo, no será grande y profesional como gnu) para clones
386(486). Esto está en funcionamiento desde abril y está comenzando a quedar
listo. Me gustaría algún comentario sobre cosas que ames/odies en minix, ya que
mi sistema operativo se parece un poco (el mismo diseño de la capa física del
sistema de archivos (por razones prácticas) entre otras cosas). Actualmente he
portado bash (1.08) y gcc (1.40), y las cosas parecen funcionar. Esto implica
que llegaré a algo práctico dentro de unos pocos meses, y quisiera saber qué
características desearía la mayoría de la gente. Cualquier sugerencia es
bienvenida, pero no prometo que pueda aplicarlas." \\

El foro al que envió ese mensaje era para discusiones sobre el sistema
operativo Minix, creado por el profesor Tanenbaum y su grupo. A inicios de
1992, cuando vio que las discusiones sobre Linux habían sobrepasado las
relativas a Minix, Tanenbaum decidió intervenir. Fue así que el 29 de enero
envió al foro un mensaje con el título "Linux es obsoleto”. Con un título como
ése, Linus no podía quedarse callado…\\

Tanenbaum afirmaba que los sistemas operativos monolíticos eran un paso atrás
en la computación; aumentaban la complejidad y disminuían su portabilidad, y
que Linux estaba demasiado orientado a la arquitectura x86. Linux reconoció que
la arquitectura micronúcleo era teóricamente superior. No obstante, argumentó
que Minix tenía errores básicos para un sistema operativo y que, debido a la
interfaz de aplicaciones de Linux, su sistema operativo era más portable que
Minix.\\

\section*{Opinión}

Al termino de la lectura del debate me surgieron algunas dudas la más
importante, creo, era en que consiste un sistema operativo micronúcleo y uno
monolitico (No he llevado operativos aun jeje). Por lo que investigué y noté que
la diferencia principal entre los micronúcleo y monoliticos es que los primeros
implementan las funcionalidades del núcleo con llamadas de sistema que se
ejecutan en como procesos servidores en espacio de usuario. Por lo tanto, las
distintas funcionalidades son separadas y son estos servidores los que las
proveen cuando son pedidas. En cambio, un núcleo monolítico implementa todas
estas funcionalidades él sólo, por lo que su complejidad se ve incrementada.\\

En lo personal pienso que la razón la tuvo Linus, ya que x86 sigue siendo, con una amplia mayoría, la arquitectura más usada en ordenadores personales. Además, en la actualidad Linux ha sido portado a muchas otras arquitecturas como MIPS, ARM, SPARC o PowerPC.



\newpage

\textbf{Bibliografía} \\

\begin{itemize}

\item \url{https://smaldone.com.ar/documentos/libros/opensources.pdf?fbclid=IwAR05n9RArmsgx3k6gxo2pTtzbIw6pUGr2QM5alJpoiTjdZx8Lx8vsZsksEk}
  
\end{itemize}

\end{document}
