\documentclass[11pt, a4paper]{report}

\usepackage[utf8]{inputenc}
\usepackage{fancyvrb}
\usepackage{enumitem}
\usepackage{hyperref}
\usepackage{multirow}
\usepackage{graphicx}

\begin{document}
\title{Publicaciones del mes de Enero}
\author{
  Carmona Mendoza Mat\'in\\
  \texttt{313075977}
}
\date{}
\maketitle

\section*{Publicaciones que seleccioné}
\begin{itemize}

\item Publicación 1.
  \begin{itemize}
  \item Systemd Security Holes Discovered.
  \item Linux Journal.
  \item 11 Enero 2019.
  \item \url{https://bit.ly/2IfKiOB}
  \end{itemize}
  
\item Publicación 2
  \begin{itemize}
  \item Cambiar la fecha de un archivo desde la consola.
  \item SoloLinux.
  \item 05 Enero 2019. 
  \item \url{https://bit.ly/2WJt8gq}
  \end{itemize}

\item Publicación 3
  \begin{itemize}
  \item ¿Es Wayland competente para la ejecución de juegos?.
  \item muyLinux.
  \item 02 Enero 2019.
  \item \url{https://bit.ly/2XMRU0v}
  \end{itemize}
\end{itemize}  

\section*{Descripción de la publicación}
Elegí la publicación número 2, Cambiar la fecha de un archivo desde la consola,
la elegí por una razón en especial, en mi curso de Logica computacional en algun
momento se me olvidó enviar una práctica que se me había pedido, y al final de
curso tenia mejorar mi promedio a toda costa, entonces le dije al ayudante que
se me había olvidado enviar una práctica pero que sí la había hecho en tiempo y
forma a lo que el ayudante me respondió, está bien, enviamela y veré la última
modificación de los archivos para comprobar que sí hiciste la práctica en
tiempo, yo sorprendido no supe qué hacer y decidí ya no enviar la práctica.
Entonces si hubiera sabido que se podia alterar la fecha a los archivos me
hubiera ido mejor. \\

El articulo explica a detalle todo el proceso para cambiar la fecha a un
archivo, comienza mostrando el comando para conocer las fechas que tiene un
archivo y sigue con el proceso para modificar la fecha del archivo en cuestión.

\section*{Conceptos que tuve que investigar}
No tuve que buscar ningun concepto, entendí muy bien el articulo.

\section*{Personas a las que recomendaría el articulo}
No es un escrito complejo entonces cualquier persona con nociones de computación
y un leve manejo de terminal podrá entenderlo así que yo lo recomiendo para
novatos.

\end{document}
