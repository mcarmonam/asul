\documentclass[11pt, a4paper]{report}

\usepackage[utf8]{inputenc}
\usepackage{fancyvrb}
\usepackage{enumitem}
\usepackage{hyperref}
\usepackage{multirow}
\usepackage{graphicx}

\begin{document}
\title{15/Marzo/2019}
\author{
  Carmona Mendoza Mat\'in\\
  \texttt{313075977}
}
\date{}
\maketitle

Hoy en clase con el ayudante continuamos revisando algunas de las opciones de
que se pueden configurar antes de compilar el kernel, entre las cuales
revisamos un módulo para procesadores Intel. También continuamos revisando como
aplicar parches al kernel, esto es útil para agregar algún modulo que no venga
por defecto en el. \\

Al Kernel le debemos que hayamos llegado tan profundo en la informática
verdadera (en la que se sabe ciertamente cómo se maneja la información en su
más bajo nivel). En la actualidad no es tan recurrente que necesitemos compilar
un kernel, aunque tiempo atrás si lo era. Hoy día esto se hace por alguna
necesidad específica o por el simple hecho de aprender cómo se hace. \\

Entre las razones existentes para compilar un kernel tenemos:

\begin{itemize}
\item Eres un desarrollador del kernel.
\item Necesitas compilar un kernel con alguna característica especial que un
  kernel oficial de tu distro no tiene.
\item Te estás preparando para corregir un bug en el kernel oficial de tu
  distro.
\item Tienes hardware que el kernel oficial de tu distro no soporta.
\end{itemize}

Existen muchas maneras de compilar el kernel (casi una por cada distribución).
Esto significa que para cada distro, ya sea basada en Red Hat, Debian o
Slackware, debemos documentarnos muy bien antes de comenzar tan importante
tarea, porque si se nos queda algún módulo excluido o incluido sin necesidad
podría generar conflictos e incluso no funcionar correctamente. \\

Todo el proceso se puede resumir en los siguientes pasos generales para todas
las distros:

\begin{itemize}
\item Obtener las fuentes del kernel.
\item Instalar las herramientas necesarias para la compilación (gcc,
  build-essentials, etc.).
\item Descomprimir las fuentes del kernel en /usr/src/
\item Configurar el kernel (ej. make menuconfig).
\item Compilar (make), compilar los módulos (make modules), instalar los
  módulos, (make modules\_install).
\item Instalar el kernel compilado (make install).
\item Crear la imagen (initrd) para el nuevo kernel y ajustar grub para que la
  encuentre.
\end{itemize}

Este proceso puede ser una excelente manera de ajustar nuestro sistema para que
soporte nuevo hardware o para hacerlo más liviano quitando, cuidadosamente,
módulos que no utilizaremos. De cualquier manera es recomendable que aparte de
esta guía busques la ayuda de personas más experimentadas si tienes una buena
razón para hacerlo, aún si lo haces por puro placer o aprendizaje.

\newpage

\textbf{Bibliografía} \\

\begin{itemize}

\item \url{https://www.muylinux.com/2010/11/10/como-compilar-el-kernel-linux-en-ubuntu-fedora-y-otras/}


\end{itemize}


\end{document}
