\documentclass[11pt, a4paper]{report}

\usepackage[utf8]{inputenc}
\usepackage{fancyvrb}
\usepackage{enumitem}
\usepackage{hyperref}
\usepackage{multirow}
\usepackage{graphicx}

\begin{document}
\title{Clase 29: 08/Marzo/2019}
\author{
  Carmona Mendoza Mat\'in\\
  \texttt{313075977}
}
\date{}
\maketitle

En la clase de hoy y parte de laboratorio seguimos con la instalción de Debian
sin entorno gráfico. En esta ocasión hicimos el proceso en una maquina virtual.

Pondré los pasos a seguir:

\begin{itemize}
\item Pon la imagen de debian y reinicia tu pc. Y seleccionas Install.

  \begin{figure}[!ht] 
    \begin{center}  
      \includegraphics[width=0.5\textwidth]{1.png} 
      \caption{} 
    \end{center} 
  \end{figure} 

\item Luego sigue el proceso agregando un dominio, si no tienes un dominio
  puedes inventar uno, sólo asegúrate de recordarlo para después agregarlo en
  otra máquina si es necesario.

  \begin{figure}[!ht] 
    \begin{center}  
      \includegraphics[width=0.5\textwidth]{2.png} 
      \caption{} 
    \end{center} 
  \end{figure}
  
\item Ahora es tiempo de particionar el disco, si no tienes idea puedes
  seleccionar el método guiado para dejar que el programa cree las particiones
  automáticamente

  \begin{figure}[!ht] 
    \begin{center}  
      \includegraphics[width=0.5\textwidth]{3.png} 
      \caption{} 
    \end{center} 
  \end{figure}
  
\item Cuando hayas terminado con las particiones el programa, el sistema
  escaneará el disco y te preguntará si quieres escanear otro disco, selecciona
  la opción NO si este es el caso.
  
  \begin{figure}[!ht] 
    \begin{center}  
      \includegraphics[width=0.5\textwidth]{4.png} 
      \caption{} 
    \end{center} 
  \end{figure}

\item En la siguiente pantalla selecciona tu país para conectarte al servidor
  optimo, luego configura un proxy si aplica en tu caso, acepta participar o no
  en el programa de usuarios y llegarás al apartado del entorno grafico SI no
  quieres un entorno grafico solo debes seleccionar el servidor que prefieras,
  web server, print server o ssh server. 
  
  \begin{figure}[!ht] 
    \begin{center}  
      \includegraphics[width=0.5\textwidth]{5.png} 
      \caption{} 
    \end{center} 
  \end{figure}

\end{itemize}

Por último, solo te hará falta elegir el lugar donde quieres instalar el GRUB Bootloader y tu PC se reiniciará con la instalación completa.

\newpage

\textbf{Bibliografía} \\

\begin{itemize}

\item \url{https://ayudalinux.com/instalar-debian-sin-entorno-grafico/}
\end{itemize}


\end{document}
