\documentclass[11pt, a4paper]{report}

\usepackage[utf8]{inputenc}
\usepackage{fancyvrb}
\usepackage{enumitem}
\usepackage{hyperref}
\usepackage{multirow}
\usepackage{graphicx}

\begin{document}
\title{Publicaciones del mes de Abril}
\author{
  Carmona Mendoza Mat\'in\\
  \texttt{313075977}
}
\date{}
\maketitle

\section*{Publicaciones que seleccioné}
\begin{itemize}

\item Publicación 1.
  \begin{itemize}
  \item Ubuntu 19.04 "Disco Dingo" Released.
  \item Linux Journal.
  \item 18 Abril 2019.
  \item \url{https://bit.ly/2WHHHRm}
  \end{itemize}
  
\item Publicación 2
  \begin{itemize}
  \item Borrar archivos temporales en Linux.
  \item SoloLinux.
  \item 30 Abril 2019. 
  \item \url{https://bit.ly/2Rgbz6t}
  \end{itemize}

\item Publicación 3
  \begin{itemize}
  \item Mejora la calidad del sonido en Linux con Xonar y Essence.
  \item muyLinux.
  \item 15 Abril 2019.
  \item \url{https://bit.ly/2wXBXsl}
  \end{itemize}
\end{itemize}  

\section*{Descripción de la publicación}
Elegí el artículo numero 3, Mejora la calidad del sonido en Linux con Xonar y
Essence por que soy una persona que le gusta escuchar música mientras está en el
ordenador y por ello he tenido varias inconvenientes cuando escucho musica. \\

En los últimos años el soporte de hardware ha mejorado mucho en GNU/Linux.
Afortunadamente, hoy en día los usuarios del sistema Open Source pueden elegir
qué GPU quieren utilizar sin miedo a que el rendimiento sea insuficiente o el
funcionamiento incorrecto gracias sobre todo el gran salto cualitativo
experimentado por Intel y AMD. Sin embargo, hay un gran frente que sigue sin
estar bien cubierto: las tarjetas de sonido dedicadas. \\

El hecho de que las tarjetas de sonido dedicadas sigan sin estar bien
soportadas en GNU/Linux tiene un culpable muy claro: Creative. Esto tiene como
consecuencia que muy pocas tarjetas internas Sound Blaster funcionan con Linux.
Pese a ello, no todo está perdido, ya que los productos de uno de sus
principales competidores, ASUS Xonar y Essence, sí están relativamente bien
soportados por Linux, y digo relativamente porque en muchos casos se requiere
de pequeñas configuraciones para que den el 100\%, aunque sí son detectadas
correctamente por el kernel desde el primer inicio.

Entonces el texto te detalla el proceso de configuración de tu tarjeta de sonido
para tener buena calidad de audio yevitar problemas futuros.



\section*{Conceptos que tuve que investigar}
Ninguno, todos los conceptos que se manejaron en el texto los tendí bien.

\section*{Personas a las que recomendaría el articulo}
Este texto lo recomiendo para personas intermedias conocedores de tarjetas de
audio y con conocimiendo sobre la terminal de Linux.

\end{document}
