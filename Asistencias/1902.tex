\documentclass[11pt, a4paper]{report}

\usepackage[utf8]{inputenc}
\usepackage{fancyvrb}
\usepackage{enumitem}

\begin{document}
\title{Clase 16: 19/Febrero/2019}
\author{
  Carmona Mendoza Mat\'in\\
  \texttt{313075977}
}
\date{}
\maketitle

En la clase de hoy vimos cigwin es una colección de herramientas
para proporcionar un comportamiento similar a los sistemas
Unix en Microsoft Windows. Su objetivo es portar software que ejecuta en
sistemas POSIX a Windows con una recompilación a partir de sus fuentes. \\

Hablamos también de la carpeta dev En esta carpeta se encuentran todos
los archivos que nos permiten interactuar con los dispositivos hardware
de nuestra PC. Por ejemplo los usb, sda (o hda) con la información de cada
uno de ellos. \\

Vimos estos comandos: \\
\\
$mount --bind /dev/ /mnt/dev/$ \\
$mount --bind /sys/ /mnt/sys/$ \\
$mount --bind /proc/ /mnt/proc/$ \\

Con esto podemos realizar el montaje. 



\end{document}
