\documentclass[11pt, a4paper]{report}

\usepackage[utf8]{inputenc}
\usepackage{fancyvrb}
\usepackage{enumitem}
\usepackage{hyperref}
\usepackage{multirow}

\begin{document}
\title{Clase 24: 04/Marzo/2019}
\author{
  Carmona Mendoza Mat\'in\\
  \texttt{313075977}
}
\date{}
\maketitle
En la clase de hoy vimos algunas razones (persnales) por la cuál utilzas
linux en lugar de Windows y viceversa. Algunas de las razones que
comentaron mis compañeros fueron:

\begin{itemize}
\item Yo utilizo Windows para jugar, es decir los juegos "pupulares" son
  lanzados principalmente para Windows entonces estoy obligado a usar
  Windows.
\item Yo utilizo Windows por que es más gráfico, por ejemplo para los
  sistemas manejadores de bases de datos en Windows cada uno de ellos
  cuenta con una interfaz gráfica y el Linux es menos gráfico.
\end{itemize}

Aqui dejamos algunas razones por las cuales puedes usar Linux:

\begin{itemize}
\item No tienes que comprar una licencia : No tienes que pagar 175 USD
  para poder adquirir una licencia, no tienes que estar buscando algún
  activador de no tener una original. Aquí todo es libre, al menos las
  distribuciones dirigidas para usuarios y no para servidores.
\item  Puedes revivir un ordenador viejito : Si tienes un ordenador con
  muy pocos recursos es posible que ya no pueda correr Windows 10, pero
  en cambio puedes encontrar una distribución que los requisitos del
  sistema sean bien mínimos, como por ejemplo 512 de RAM.
\item Casi todos los programas de uso básico disponibles : Podemos
  encontrar casi todos los programas de uso diario, me refiero a un
  editor de texto, hojas de cálculos, edición de imágenes, navegadores,
  entre otras herramientas que solemos usar.
\item Puedes manejar el sistema desde la linea de comandos : Ya se que
  esto no es para todos, pero una ventaja es que tu puedes manejar
  prácticamente todo el sistema sin tener una interfaz gráfica, puedes
  instalar, desinstalar programas desde esta modalidad, no solamente eso,
  sino que puedes hacer tantas cosas como no tienes idea.
\item Puedes probarlos sin instalarlos : Puedes ejecutar una distribución
  de linux sin la necesidad de tenerlo que instalar, a diferencia de
  Windows que no es posible. Ejecutar una distribución desde un live cd o
  una memoria USB es muy sencillo.
\item Muchas distribuciones : Si nos ponemos a investigar no solamente
  existe ubuntu, sino que hay una gran cantidad que puedes, ir
  probándolas poco a poco, a lo mejor no en tu ordenador principal, pero
  si en uno secundario, por la misma razón de las particiones.
\item Puedes vivir sin preocupaciones de virus : Windows desde siempre ha
  tenido ese problema de virus, y no es para menos, yo he tenido varias
  veces que formatear alguno de mis ordenadores debido a que ya no habido
  más que hacer. En cambio linux no existe este tipo de cosas, no hay
  porque tener un antivirus, esto trate una ventaja para nosotros, como
  por ejemplo menos uso de recursos.
\end{itemize}


Despues vimos qué distribuciones de Linux eran las más usadas haciendo uso
de la página distrowatch. Teniendo como top ten
(de estos últimos 6 meses) a:

\begin{itemize}
\item Manjaro.
\item MX Linux.
\item Mint.
\item Elementary.
\item Ubuntu.
\item Debian.
\item Fedora.
\item Solus.
\item openSUSE.
\item Zorin. 
\end{itemize}


\newpage

\textbf{Bibliografía} \\

\begin{itemize}

\item  \url{https://blog.desdelinux.net/10-razones-usar-linux/}
\item  \url{https://distrowatch.com/?language=ES}  
\end{itemize}


\end{document}
