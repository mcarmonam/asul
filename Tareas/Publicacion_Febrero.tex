\documentclass[11pt, a4paper]{report}

\usepackage[utf8]{inputenc}
\usepackage{fancyvrb}
\usepackage{enumitem}
\usepackage{hyperref}
\usepackage{multirow}
\usepackage{graphicx}

\begin{document}
\title{Publicaciones del mes de Febrero}
\author{
  Carmona Mendoza Mat\'in\\
  \texttt{313075977}
}
\date{}
\maketitle

\section*{Publicaciones que seleccioné}
\begin{itemize}

\item Publicación 1.
  \begin{itemize}
  \item Ubuntu Posts Security Notice for systemd Vulnerability and Applications
    Open for Outreachy Summer 2019 Internships.
  \item Linux Journal.
  \item 19 Febrero 2019.
  \item \url{https://bit.ly/31BOcJ8}
  \end{itemize}
  
\item Publicación 2
  \begin{itemize}
  \item Instalar nuevas fuentes en Linux.
  \item SoloLinux.
  \item 02 Enero 2019. 
  \item \url{https://bit.ly/2XORjeD}
  \end{itemize}

\item Publicación 3
  \begin{itemize}
  \item ¿Mal rendimiento con WSL? Instala Linux, pero no desactives el
    antivirus.
  \item muyLinux.
  \item 14 Febrero 2019.
  \item \url{}
  \end{itemize}
\end{itemize}  

\section*{Descripción de la publicación}
Elegí el artículo número 2, instalar nuevas fuentes en Linux, porque como
mencioné en un reporte pasado a mí me "preocupa" mucho la apariencia de mi
computardor, a esto me refiero con los temas de las ventanas. el tema de los
íconos, el escritorio, etc.. Por defecto Ubuntu tiene una lista predeterminada
de fuentes, son variadas pero no son de mi agrado, siempre creí que no se podian
agregar más pero este articulo me sirvió para darme cuenta que estaba
equivocado. \\

Este articulo comienza explicandote la historia del problema de las "fuentes"
con el proyecto Core fonts for the Web el cual que liderado por Microsoft
pretendía estandarizar un conjunto de tipos de letra para su uso en la web. \\

Después de manera clara y detallada te muestra los pasos que debes seguir para
instalar las nuevas fuentes. Incluso puedes agregar fuentes de Microsoft y de
Google a tu Linux.

\section*{Conceptos que tuve que investigar}
Es un texto que solo habla de historia y uso de comandos nada complejos por lo
que entendí el texto a la perfección sin tener que buscar nada.

\section*{Personas a las que recomendaría el articulo}
Lo recomiendo para novatos y especialmente para personas que les gusta
personalizar sus ordenadores.

\end{document}
