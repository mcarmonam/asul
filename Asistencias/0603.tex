\documentclass[11pt, a4paper]{report}

\usepackage[utf8]{inputenc}
\usepackage{fancyvrb}
\usepackage{enumitem}
\usepackage{hyperref}
\usepackage{multirow}

\begin{document}
\title{Clase 27: 06/Marzo/2019}
\author{
  Carmona Mendoza Mat\'in\\
  \texttt{313075977}
}
\date{}
\maketitle

En la clase de hoy continuamos con lo que habíamos dejado pendiente, continuar
con la instalación de Debian con la imagen pequeña. El profesor nos dijo que lo
que habíamos hecho la clase pasada estaba mal, es decir descargamos archivos
diferentes. Entonces descargamos la mini.iso de arranque por red de Debian e
hicimos los mismo pasos de la clase pasada.  \\

\textbf{Tipos de arranque por red PXE.} \\

\begin{itemize}
\item Modalidad Kernel (sólo BIOS) : Durante el arranque, el programa de
  precarga inicia automáticamente un kernel de despliegue del SO en el destino
  para descargar e instalar el sistema operativo Windows o Linux según la
  información almacenada en la base de datos del Servidor de despliegue de
  sistema operativo. Este kernel es una mini sistema operativo que establece
  contacto con el Servidor de despliegue de sistema operativo y ejecuta el
  despliegue en la máquina de destino. La ventaja de utilizar este tipo de
  arranque es que la descarga del motor de despliegue es más rápida que la
  descarga del mismo en la modalidad sin kernel.
\item Modalidad sin kernel (BIOS) : Durante el arranque, el programa de
  precarga inicia automáticamente la herramienta de Linux PXE en el destino
  para descargar e instalar el sistema operativo Windows o Linux según la
  información almacenada en la base de datos del Servidor de despliegue de
  sistema operativo. Esta herramienta, que forma parte de la suite SysLinux,
  permite que el kernel de Windows o Linux se inicie sin utilizar ningún
  programa del Servidor de despliegue de sistema operativo. 
\item Modalidad sin kernel (UEFI) : Durante el arranque, el programa de
  precarga inicia automáticamente la herramienta conmutador (switcher) de UEFI
  en el destino para descargar e instalar el sistema operativo Windows según la
  información almacenada en la base de datos del Servidor de despliegue de
  sistema operativo. Esta herramienta permite a Windows iniciarse sin utilizar
  ningún programa del Servidor de despliegue de sistema operativo.  
\end{itemize}

Repetimos los mismos pasos que la clase pasado, entonces entando en el grub
vimos los siguientes comandos.

\begin{itemize}
\item boot : Arranca el sistema operativo o gestor de encadenamiento que se ha
  cargado.
\item chainloader </ruta/a/archivo> : Carga el archivo especificado como gestor
  de encadenamiento. Si el archivo está ubicado en el primer sector de la
  partición especificada, puede utilizar la notación de lista de bloques, +1,
  en vez del nombre del archivo.
\item displaymem : Muestra el uso actual de memoria, en función de la
  información de la BIOS. Esto es útil si no está seguro de la cantidad de RAM
  que tiene un sistema y todavía tiene que arrancarlo.
\item initrd </ruta/a/initrd> : Le permite especificar un disco RAM inicial
  para utilizarlo al arrancar. Es necesario un initrd cuando el kernel necesita
  ciertos módulos para poder arrancar adecuadamente, tales como cuando la
  partición se formatea con el sistema de archivos ext3.
\item root (<tipo-dispositivo><numero-dispositivo>,<particion>) : Configura la
  partición raíz para GRUB, tal como (hd0,0) y monta la partición.
\end{itemize}

\newpage

\textbf{Bibliografía} \\

\begin{itemize}

\item  \url{https://www.debian.org/CD/netinst/}
\item \url{https://www.ibm.com/support/knowledgecenter/es/SS3HLM_7.1.1.16/com.ibm.tivoli.tpm.osd.doc_7.1.1.16/deploy/cosd_kernelfree.htm}
\item \url{https://web.mit.edu/rhel-doc/4/RH-DOCS/rhel-rg-es-4/s1-grub-commands.html}
\end{itemize}


\end{document}
