\documentclass[11pt, a4paper]{report}

\usepackage[utf8]{inputenc}
\usepackage{fancyvrb}
\usepackage{enumitem}
\usepackage{hyperref}
\usepackage{multirow}
\usepackage{graphicx}

\begin{document}
\title{21/Marzo/2019}
\author{
  Carmona Mendoza Mat\'in\\
  \texttt{313075977}
}
\date{}
\maketitle

Hoy hablamos sobre las capas del modelo OSI el cual se desarrolló allá por 1984
la organización ISO (International Organization for Standarization). Este
estándar perseguía el ambicioso objetivo de conseguir interconectar sistema de
procedencia distinta para que esto pudieran intercambiar información sin ningún
tipo de impedimentos debido a los protocolos con los que estos operaban de
forma propia según su fabricante. \\

El modelo OSI está conformado por 7 capas o niveles de abstracción. Cada uno de
estos niveles tendrá sus propias funciones para que en conjunto sean capaces de
poder alcanzar su objetivo final. Precisamente esta separación en niveles hace
posible la intercomunicación de protocolos distintos al concentrar funciones
específicas en cada nivel de operación. \\

Otra cosa que debemos tener muy presente es que el modelo OSI no es la
definición de una topología ni un modelo de red en sí mismo. Tampoco especifica
ni define los protocolos que se utilizan en la comunicación, ya que estos están
implementados de forma independiente a este modelo. Lo que realmente hace OSI
es definir la funcionalidad de ellos para conseguir un estándar. \\

Capa 1: Física\\
Este nivel se encarga directamente de los elementos físicos de la conexión.
Gestiona los procedimientos a nivel electrónico para que la cadena de bits de
información viaje desde el transmisor al receptor sin alteración alguna. \\

Capa 2: Enlace de datos \\
Este nivel se encarga de proporcionar los medios funcionales para establecer la
comunicación de los elementos físicos. Se ocupa del direccionamiento físico de
los datos, el acceso al medio y especialmente de la detección de errores en la
transmisión. \\

Capa 3: Red \\
Esta capa se encarga de la identificación del enrutamiento entre dos o más
redes conectadas. Este nivel hará que los datos puedan llegar desde el
transmisor al receptor siendo capaz de hacer las conmutaciones y
encaminamientos necesarios para que el mensaje llegue. Debido a esto es
necesario que esta capa conozca la topología de la red en la que opera. \\

Capa 4: Transporte \\
Este nivel se encarga de realizar el transporte de los datos que se encuentran
dentro del paquete de transmisión desde el origen al destino. Esto se realiza
de forma independiente al tipo de red que haya detectado el nivel inferior. La
unidad de información o PDU antes vista, también le llamamos Datagrama si
trabaja con el protocolo UPD orientado al envío sin conexión, o Segmento, si
trabaja con el protocolo TCP orientado a la conexión. \\

Capa 5: Sesión \\
Mediante este nivel se podrá controlar y mantener activo el enlace entre las
máquinas que están transmitiendo información. De esta forma se asegurará que
una vez establecida la conexión, esta e mantengas hasta que finalice la
transmisión. \\

Capa 6: Presentación \\
Como su propio nombre intuye, esta capa se encarga de la representación de la
información transmitida. Asegurará que los datos que nos llegan a los usuarios
sean entendibles a pesar de los distintos protocolos utilizados tanto en un
receptor como en un transmisor. Traducen una cadena de caracteres en algo
entendible, por así decirlo. \\

Capa 7: Aplicación \\
Este es el último nivel, y en encargado de permitir a los usuarios ejecutar
acciones y comandos en sus propias aplicaciones como por ejemplo un botón para
enviar un email o un programa para enviar archivos mediante FTP. Permite
también la comunicación entre el resto de capas inferiores. \\

\newpage

\textbf{Bibliografía} \\

\begin{itemize}

\item \url{https://www.profesionalreview.com/2018/11/22/modelo-osi/}


\end{itemize}


\end{document}
