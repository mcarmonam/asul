\documentclass[11pt, a4paper]{report}

\usepackage[utf8]{inputenc}
\usepackage{fancyvrb}
\usepackage{enumitem}
\usepackage{hyperref}
\usepackage{multirow}
\usepackage{graphicx}

\begin{document}
\title{Publicaciones del mes de Marzo}
\author{
  Carmona Mendoza Mat\'in\\
  \texttt{313075977}
}
\date{}
\maketitle

\section*{Publicaciones que seleccioné}
\begin{itemize}

\item Publicación 1.
  \begin{itemize}
  \item Linux Kernel 5.0 Is Officially Out.
  \item Linux Journal.
  \item 04 Marzo 2019.
  \item \url{https://bit.ly/2wUPjpp}
  \end{itemize}
  
\item Publicación 2
  \begin{itemize}
  \item Aumentar el valor en max\_allowed\_packet o wait\_timeout.
  \item SoloLinux.
  \item 20 Marzo 2019. 
  \item \url{https://bit.ly/2ICY1Ok}
  \end{itemize}

\item Publicación 3
  \begin{itemize}
  \item Hoy hace 25 años que se lanzó Linux 1.0.
  \item muyLinux.
  \item 14 Marzo 2019.
  \item \url{https://bit.ly/2ZoMZmH}
  \end{itemize}
\end{itemize}  

\section*{Descripción de la publicación}
Escogí el articulo 3, Hoy hace 25 años que se lanzó Linux 1.0, porque me pareció
bastante interesante y porque no sabia la historia que tiene el sistema
operativo que llevo usando desde hace 4 años. \\

El artículo habla tal cual de la historia de linux, nos cuenta que el 14 de
marzo del año de 1994 Linus Torvalds da a conocer al mundo el proyecto personal
que desarrollaba "solo por afición", pero también para ahorrarse el excesivo
coste de las licencias de los Unix de la época. \\

Siguiendo con la historia, Linux 2.0 llegaría en junio de 1996 (la versión más
extensa en el tiempo por mucho, casi tres lustros, aunque con cambios
importantes en su haber; de esta experiencia viene el rechazo de Torvalds a las
numeraciones largas), Linux 3.0 en en julio de 2011, Linux 4.0 en abril de 2015.
\\

\section*{Conceptos que tuve que investigar}
Es un texto de meramente historia por lo que es fácil entender así que no tuve
que buscar algun concepto. 

\section*{Personas a las que recomendaría el articulo}
Este articulo se lo recomiento a todas las personas ya que relata la historia
de uno de los sistemas operativos más influyentes en la actualidad. 

\end{document}
