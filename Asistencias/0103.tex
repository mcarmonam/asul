\documentclass[11pt, a4paper]{report}

\usepackage[utf8]{inputenc}
\usepackage{fancyvrb}
\usepackage{enumitem}
\usepackage{hyperref}
\usepackage{multirow}

\begin{document}
\title{Clase 23: 01/Marzo/2019}
\author{
  Carmona Mendoza Mat\'in\\
  \texttt{313075977}
}
\date{}
\maketitle

En la clase de hoy hablamos sobre Xorg siendo un servidor X con todas las
funciones que fue diseñado originalmente para sistemas operativos UNIX y
similares a UNIX que se ejecutan en hardware Intel x86. Ahora se ejecuta en una
amplia gama de plataformas de hardware y sistemas operativos. \\

Xorg opera bajo una amplia gama de sistemas operativos y plataformas de
hardware. La arquitectura Intel x86 (IA32) es la plataforma de hardware más
ampliamente admitida. Otras plataformas de hardware incluyen Compaq Alpha,
Intel IA64, SPARC y PowerPC. Los sistemas operativos más soportados son los
sistemas gratuitos / OpenSource como UNIX, como Linux, FreeBSD, NetBSD y
OpenBSD. Los sistemas operativos comerciales UNIX como Solaris (x86) y UnixWare
también son compatibles. Otros sistemas operativos compatibles incluyen LynxOS
y GNU Hurd. Darwin y Mac OS X son compatibles con el servidor XDarwin (1) X.
Win32 / Cygwin es compatible con el servidor XWin X. \\

Xorg soporta conexiones hechas usando los siguientes flujos de bytes confiables:

\begin{itemize}
\item Local : En la mayoría de las plataformas, el tipo de conexión "Local" es
  un socket de dominio UNIX. En algunas plataformas del Sistema V, los tipos de
  conexión "locales" también incluyen tuberías STREAMS, tuberías con nombre y
  algunos otros mecanismos.
\item TCP / IP : Xorg escucha en el puerto 6000+ n , donde n es el número de
  pantalla. Este tipo de conexión se puede desactivar con la opción -nolisten
  (consulte la página de manual de Xserver (1) para obtener más información).
\end{itemize}

Xorg admite varios mecanismos para suministrar / obtener parámetros de configuración y tiempo de ejecución:

\begin{itemize}
\item vt XX : XX especifica el número de dispositivo del terminal virtual que
  Xorg utilizará. Sin esta opción, Xorg elegirá la primera Terminal virtual
  disponible que pueda localizar. Esta opción se aplica solo a plataformas como
  Linux, BSD, SVR3 y SVR4, que tienen soporte de terminal virtual.
\item pixmap24 : Configure el formato de mapa de píxeles interno para 24
  píxeles de profundidad a 24 bits por píxel. El valor predeterminado suele ser
  de 32 bits por píxel. Normalmente hay pocas razones para usar esta opción. A
  algunas aplicaciones cliente no les gusta este formato de mapa de píxeles,
  aunque es un formato perfectamente legal. Esto es equivalente a la opción de
  archivo xx.conf (5x) de Pixmap.
\item lasilk : Desactiva el soporte de Silken Mouse.
\item flipPixels : Intercambia los valores predeterminados por los píxeles en
  blanco y negro.
\item fbbpp n : Establece el número de bits de framebuffer por píxel. Solo
  debes configurar esto si estás seguro de que es necesario; normalmente el
  servidor puede deducir el valor correcto de -depth anterior. Útil si desea
  ejecutar una configuración de profundidad 24 con un framebuffer de 24 bpp en
  lugar del (probablemente predeterminado) 32 bpp framebuffer (o viceversa).
  Los valores legales son 1, 8, 16, 24, 32. No todos los controladores admiten
  todos los valores.
\item bgamma valor : Establecer la corrección de gamma azul. el valor debe
  estar entre 0.1 y 10. El valor predeterminado es 1.0. No todos los
  controladores soportan esto. Consulte también las opciones -gamma , -rgamma
  y -ggamma .
\end{itemize}

Después hablamos de Autotools siendo es un conjunto de herramientas producido
por el proyecto GNU. Estas herramientas están diseñadas para ayudar a crear
paquetes de código fuente portable a varios sistemas Unix. El GNU Build System
forma parte de la cadena de herramientas de GNU y se usa mucho para desarrollar
software libre. Aunque las herramientas que contiene el GNU Build System son
GPL no existe ninguna restricción para crear software portable no libre con él. \\

Autoconf procesa los archivos
configure.in o configure.ac (aunque se recomienda
usar configure.ac1​). Cuando ejecuta el script de configuración también puede
procesar otros archivos como Makefile.in para producir como salida un archivo
Makefile. \\

Autoconf se usa para intentar salvar las diferencias que existen entre
distintos tipos de Unix. Por ejemplo, algunos sistemas Unix pueden tener
funcionalidades que no existen o no funcionan en otros sistemas. Autoconf puede
detectar ese problema y busca la forma de solucionarlo. La salida de Autoconf
es un script denominado configure. Autoconf incluye la herramienta Autoheader
que se usa para manejar los archivos de cabecera de C. \\

También se mencionó el protocolo Wayland que es un protocolo de servidor
gráfico y una biblioteca para GNU/Linux que implementa este protocolo. \\​

Wayland proporciona un método para que los gestores de composición de ventanas
se comuniquen directamente con las aplicaciones y el hardware de vídeo. Se
espera que también sea posible la comunicación con hardware de entrada usando
otras bibliotecas. Las aplicaciones renderizan los gráficos en sus propios
buffers y el gestor de ventanas se convierte en el servidor gráfico, haciendo
una composición con esos buffers para formar la visualización en pantalla de
las ventanas de las aplicaciones. Este es un enfoque más simple y más eficiente
que usar un gestor de composición de ventanas con el X Window System. \\

Los gestores de composición de ventanas existentes, como KWin y Mutter, se
espera que implementen soporte para Wayland de forma directa, para convertirse
en compositores Wayland/servidores gráficos.

\newpage

\textbf{Bibliografía} \\

\begin{itemize}

\item  \url{https://wiki.archlinux.org/index.php/xorg}
\item  \url{https://askubuntu.com/questions/430706/installing-autotools-autoconf}  
\item  \url{https://wayland.freedesktop.org}
\end{itemize}


\end{document}
