\documentclass[11pt, a4paper]{report}

\usepackage[utf8]{inputenc}
\usepackage{fancyvrb}
\usepackage{enumitem}
\usepackage{hyperref}
\usepackage{multirow}

\begin{document}
\title{Clase 22: 27/Febrero/2019}
\author{
  Carmona Mendoza Mat\'in\\
  \texttt{313075977}
}
\date{}
\maketitle

En la clase de hoy hablamos sobre modulos definiendo módulo como
fragmentos de código, abierto o cerrado, que nosotros podemos añadir o
quitar del Kernel con el fin de añadir o quitar una funcionalidad. \\

Observamos algunas funcionalidades de los módulos:

\begin{itemize}
\item Usar los drivers privativos de nuestra tarjeta gráfica.
\item Registrar las temperaturas de componentes de nuestro ordenador.
\item Que los ventiladores de nuestro ordenador sean gestionados por el
  software creado por el fabricante de nuestro ordenador.
\item Hacer funcionar nuestro tarjeta de red wifi.
\end{itemize}

Entonces básicamente los módulos del kernel nos permiten hacer funcionar
nuestro hardware, o en el caso que ya funcione podemos hacer que lo haga
de forma más eficiente. \\

Los módulos del kernel son archivos terminados con la extensión .ko que
se almacenan en la ubicación /lib/modules/versión\_del\_kernel. \\

Por lo tanto ejecutando el siguiente comando en la terminal podemos
averiguar la totalidad de módulos que tenemos disponibles. \\

ls -R /lib/modules/\$(uname -r) \\

Para obtener una lista de la totalidad de módulos del kernel que estamos
usando tenemos que ejecutar el siguiente comando en la terminal: \\

lsmod \\

Este comando nos mostrará la siguiente información:

\begin{table}[htbp]
\begin{center}
\begin{tabular}{|l|l|}
\hline
Campo & Información mostrada \\
\hline \hline
Module & Los nombres de los módulos que están activos. \\ \hline
Size & El tamaño que ocupa cada uno de los módulos cargados en la
memoria
\\ \hline
Used & Indica si un módulo está siendo usado por otros módulos \\ \hline
\end{tabular}
\end{center}
\end{table} 

\textbf{Cargar un módulo con sus dependencias} \\

Cargar un módulo en el kernel conjuntamente con sus dependencias es muy
sencillo. Pongamos el caso que queremos cargar el módulo vfat.\\

El primer paso a realizar es comprobar si el módulo está cargado. Para
ello ejecutamos el comando lsmod | grep seguido del nombre del módulo:\\

lsmod | grep vfat \\

Si el comando no nos proporciona ningún resultado quiere decir que el
módulo no está cargado y por lo tanto lo podemos cargar sin problema
alguno. \\

Seguidamente actualizamos la base de datos de dependencias de módulos
ejecutando el siguiente comando en la terminal: \\

sudo depmod \\

Finalmente cargamos el módulo ejecutando el comando sudo modprobe seguido
del nombre del módulo: \\

sudo modprobe -v vfat \\

Al ejecutar el comando se cargará el módulo vfat juntamente con todas sus
dependencias en el kernel de nuestro equipo. \\

Como hemos visto anteriormente hay módulos que permiten introducir
parámetros de configuración cuando los cargamos. Para conocer los
parámetros que se pueden usar consulten el apartado obtener información
de los módulos del kernel. \\

Modprobe tiene multitud de opciones para su ejecución. Para obtener
información adicional sobre este comando abran una terminal y ejecuten
el comando man modprobe.\\
\\
\textbf{Deshabilitar un módulo junto con sus dependencias} \\

Si después de cargar el módulo vfat nos arrepentimos y queremos
deshabilitarlo tenemos que seguir los siguientes pasos: \\

En la terminal ejecutamos el comando sudo modprobe -rv seguido del nombre
del módulo que queremos descargar. Por lo tanto para descargar el módulo
vfat ejecutamos el siguiente comando en la terminal: \\

sudo modprobe -rv vfat \\

Al ejecutar este comando se descargará el módulo vfat y todas sus
dependencias del Kernel.
\\
\textbf{Lista negra} \\

Puede darse el caso que existan 2 módulos activos realizando la misma
función y esto puede generar conflictos. \\

Un caso típico y conocido es tener activados los drivers privativos de
nvidia conjuntamente con los libres. \\

Para solucionar este punto disponemos de 2 soluciones: \\

Descargar uno de los 2 módulos. Para ello hay que seguir las
instrucciones descritas en el apartado anterior. \\

Poner a la lista negra uno de los 2 módulos. De esta forma el módulo que
está en la lista negra no se cargará nunca. \\

Para aplicar la segunda solución abrimos una terminal y ejecutamos el
siguiente comando: \\

sudo nano /etc/modprobe.d/blacklist.conf \\

Una vez se abra el editor de textos escribiremos el text blacklist
seguido del nombre del módulo a bloquear. Por lo tanto para poner el
módulo nouveau a la lista negra escribimos:\\

blacklist nouveau \\

A continuación guardamos los cambios y cerramos el fichero. Si ninguno de
los módulos activos depende de nouveau, la próxima vez que arranquemos el
ordenador no se cargará el módulo nouveau. \\

Si queremos bloquear de forma absoluta el módulo nouveau podemos
sustituir el comando blacklist nouveau por: \\

install nouveau /bin/false




\newpage

\textbf{Bibliografía} \\

\begin{itemize}

\item  \url{https://geekland.eu/gestionar-modulos-del-kernel-linux/}

\end{itemize}


\end{document}
