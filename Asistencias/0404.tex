\documentclass[11pt, a4paper]{report}

\usepackage[utf8]{inputenc}
\usepackage{fancyvrb}
\usepackage{enumitem}
\usepackage{hyperref}
\usepackage{multirow}
\usepackage{graphicx}

\begin{document}
\title{Clase 46: 04/Abril/2019}
\author{
  Carmona Mendoza Mat\'in\\
  \texttt{313075977}
}
\date{}
\maketitle

Hoy tocó clase con el ayudante y hablamos sobre bootstrapping siendo un término
utilizado para describir el arranque, o proceso de inicio de cualquier
ordenador. Suele referirse al programa que arranca un sistema operativo como
por ejemplo GRUB, LiLo (utilizados en sistemas GNU/Linux, por ejemplo), BCD o
NTLDR (utilizados en sistemas Windows). Se ejecuta tras el proceso POST
(power-on-self-test) del BIOS. También es llamado Boot loader (cargador de
inicio). \\

El proceso de arranque es el siguiente (con el equipo encendido):

\begin{itemize}
\item Se ejecuta el POST, que se encuentra en la dirección F000:FFF0, que
  pertenece al ROM-BIOS, destinada a realizar una serie de tests e
  inicializaciones de los componentes electrónicos conectados (hardware).
\item Se carga del disco primario el primer sector (cilindro 0, cabeza 0,
  sector 1) en la dirección 0000:7C00 (7C00 lineal).
\item Se comprueba que contenga código válido (debe estar firmado con los
  valores 0x55 y 0xAA en bytes de las posiciones 511 y 512 respectivamente), en
  cuyo caso se salta a esa dirección (a la que apunta CS:IP).
\end{itemize}

Despues de eso hablamos soble MMU, la verdad no tenía mucho conocimiento sobre
este tema pero aprendí que es responsable de gestionar la memoria del sistema. \\

Este componente sirve como una especie de caché entre la CPU y el sistema de
memoria. Las funciones que realiza la MMU se puede dividir en tres cosas, que
son la gestión de la memoria hardware, la del sistema operativo y la de
aplicación. Aunque esta unidad de gestión puede ser un componente separado de
los demás sistemas, normalmente se encuentra integrado en la CPU. \\

Por norma general, el hardware asociado con la unidad de gestión de memoria
incluye la memoria RAM y otros cachés de memoria. La RAM es un componente
físico de almacenamiento que se localiza en el disco duro, y es la zona de
almacenamiento principal donde los datos son leídos y escritos. La memoria de
caché se usa para guardar copias de ciertos datos de la memoria principal.  La
CPU accede a esta información que se encuentra en la memoria de caché, lo cual
acelera el todo el proceso. Cuando la memoria física se queda sin espacio, el
ordenador automáticamente usa memoria virtual del disco duro para hacer
funcionar el programa solicitado. \\


La MMU asigna memoria del sistema operativo a varias aplicaciones. En la unidad
de procesamiento central se encuentra el área de direccionamiento virtual, la
cual está compuesta de un rango de direcciones que se dividen en páginas. Las
páginas son bloques secundarios que son iguales en tamaño. El proceso de
paginación automática permite que el sistema operativo use espacio de
almacenamiento repartido en el disco duro. \\

También hablamos sobre DMA la cual permite a cierto tipo de componentes de una
computadora acceder a la memoria del sistema para leer o escribir
independientemente de la unidad central de procesamiento (CPU). Muchos sistemas
hardware utilizan DMA, incluyendo controladores de unidades de disco, tarjetas
gráficas y tarjetas de sonido. DMA es una característica esencial en todos los
ordenadores modernos, ya que permite a dispositivos de diferentes velocidades
comunicarse sin someter a la CPU a una carga masiva de interrupciones. \\

Una transferencia DMA consiste principalmente en copiar un bloque de memoria de
un dispositivo a otro. En lugar de que la CPU inicie la transferencia, la
transferencia se lleva a cabo por el controlador DMA. Un ejemplo típico es
mover un bloque de memoria desde una memoria externa a una interna más rápida.
Tal operación no ocupa al procesador y, por ende, éste puede efectuar otras
tareas. Las transferencias DMA son esenciales para aumentar el rendimiento de
aplicaciones que requieran muchos recursos. \\





\newpage

\textbf{Bibliografía} \\

\begin{itemize}

\item \url{https://es.wikipedia.org/wiki/Bootstrapping_(informática)}
\item \url{http://www.ordenadores-y-portatiles.com/mmu.html}
\item \url{https://conceptosarquitecturadecomputadoras.wordpress.com/acceso-directo-a-memoria-dma/}
\end{itemize}


\end{document}
