\documentclass[11pt, a4paper]{report}

\usepackage[utf8]{inputenc}
\usepackage{fancyvrb}
\usepackage{enumitem}
\usepackage{hyperref}
\usepackage{multirow}

\begin{document}
\title{Clase 28: 07/Marzo/2019}
\author{
  Carmona Mendoza Mat\'in\\
  \texttt{313075977}
}
\date{}
\maketitle

En la clase de hoy vimos los compresores de archivos mencionando algunos de
ellos así como sus carácteristicas.

La compresión pretende, en principio, transferir o almacenar la misma
información empleando la menor cantidad de espacio. Esto permite ahorrar
espacio de almacenamiento y disminuir el tiempo en la transferencia de datos. \\

Una compresión puede ser con pérdida de información/calidad (generalmente para
las imágenes y sonidos), o sin pérdida de información (para archivos o
información que no debe ser degradada, como documentos de texto). Ver
compresión con pérdida de datos y compresión sin pérdida de datos. \\

La compresión de uno o más ficheros en paquetes (zip, rar, pak, arj, etc.) no
solo suele resultar en un ahorro de espacio en disco, sino que mejora la
portabilidad de múltiples archivos. Al descomprimirse estos paquetes, se
obtiene exactamente la misma información que la original. Además los paquetes
pueden partirse en distintos volúmenes.\\

La compresión con pérdida de datos hace referencia a otro tipo de compresión utilizado generalmente para reducir el tamaño de videos, música e imágenes. En esta compresión se elimina cierta cantidad de información básica de la original, pero, en general, esa eliminación de datos es tolerable o casi imperceptible al ojo o al oído humano. \\

Tanto con pérdida de datos o sin pérdida, la información se comprime a través
de un compresor. \\

Para la compresión de datos se utilizan complejos algoritmos para descubrir
redundancia de datos, datos similares entre sí, etc. Algunos algoritmos
conocidos de compresión son el RLE, Huffman, LZW, etc. \\

Algunos compresores son:

\begin{itemize}
\item WinZip
\item WinRar
\item WinAce
\end{itemize}

En clase también hablamos sobre direcciones ipv4 e ipv6. \\

Una dirección IP es como un número telefónico o una dirección de una calle.
Cuando te conectas a Internet, tu dispositivo (computadora, teléfono celular,
tableta) es asignado con una dirección IP, así como también cada sitio que
visites tiene una dirección IP. El sistema de direccionamiento que hemos usado
desde que nació Internet es llamado IPv4, y el nuevo sistema de
direccionamiento es llamado IPv6. La razón por la cual tenemos que reemplazar
el sistema IPv4 (y en última instancia opacarlo) con el IPv6 es porque Internet
se está quedando sin espacio de direcciones IPv4, e IPv6 provee una
exponencialmente larga cantidad de direcciones IP... Veamos los números:

\begin{itemize}  
\item Total de espacio IPv4: 4,294,967,296 direcciones.
\item Total de espacio IPv6: 340,282,366,920,938,463,463,374,607,431,768,211,456 direcciones.
\end{itemize}

Incluso diciendo que IPv6 es "exponencialemente largo" realmente no se compara
en diferencia de tamaños. \\
\\
\textbf{Direcciones IPv4} \\

Para entender el por que el espacio de direcciones IPv4 es limitado a 4.3 mil
millones de direcciones, podemos descomponer una dirección IPv4. Una dirección
IPv4 es un número de 32 bits formado por cuatro octetos (números de 8 bits) en
una notación decimal, separados por puntos. Un bit puede ser tanto un 1 como un
0 (2 posibilidades), por lo tanto la notación decimal de un octeto tendría 2
elevado a la 8va potencia de distintas posibilidades (256 de ellas para ser
exactos). Ya que nosotros empezamos a contar desde el 0, los posibles valores
de un octeto en una dirección IP van de 0 a 255. \\

Ejemplos de direcciones IPv4: 192.168.0.1, 66.228.118.51, 173.194.33.16 \\

Si una dirección IPv4 está hecha de cuatro secciones con 256 posibilidades en
cada sección, para encontrar el número de total de direcciones IPv4, solo debes
de multiplicar 256*256*256*256 para encontrar como resultado 4,294,967,296
direcciones. Para ponerlo de otra forma, tenemos 32 bits entonces, 2 elevado a
la 32va potencia te dará el mismo número obtenido. \\
\\
\textbf{Direcciones IPv6} \\

Las direcciones IPv6 están basadas en 128 bits. Usando la misma matemática
anterior, nosotros tenemos 2 elevado a la 128va potencia para encontrar el
total de direcciones IPv6 totales, mismo que se mencionó anteriormente. Ya que
el espacio en IPv6 es mucho mas extenso que el IPv4 sería muy difícil definir
el espacio con notación decimal... se tendría 2 elevado a la 32va potencia en
cada sección.

Para permitir el uso de esa gran cantidad de direcciones IPv6 más fácilmente,
IPv6 está compuesto por ocho secciones de 16 bits, separadas por dos puntos
(:). Ya que cada sección es de 16 bits, tenemos 2 elevado a la 16 de
variaciones (las cuales son 65,536 distintas posibilidades). Usando números
decimales de 0 a 65,535, tendríamos representada una dirección bastante larga,
y para facilitarlo es que las direcciones IPv6 están expresadas con notación
hexadecimal (16 diferentes caracteres: 0-9 y a-f). \\

Ejemplo de una dirección IPv6: 2607 : f0d0 : 4545 : 3 : 200 : f8ff : fe21 : 67cf
\\

que sigue siendo una expresión muy larga pero es mas manejable que hacerlo con
alternativas decimales.



\newpage

\textbf{Bibliografía} \\

\begin{itemize}

\item \url{http://wwwcompriar.blogspot.com}
\item \url{http://ipv6.mx/index.php/component/content/article/189-ipv4-vs-ipv6-icual-es-la-diferencia}
\end{itemize}


\end{document}
