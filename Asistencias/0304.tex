\documentclass[11pt, a4paper]{report}

\usepackage[utf8]{inputenc}
\usepackage{fancyvrb}
\usepackage{enumitem}
\usepackage{hyperref}
\usepackage{multirow}
\usepackage{graphicx}

\begin{document}
\title{Clase 45: 03/Abril/2019}
\author{
  Carmona Mendoza Mat\'in\\
  \texttt{313075977}
}
\date{}
\maketitle

Este era el último día para terminar el examen: \\

Despues de lo que hicimos el día de ayer lanzamos una terminal en el entorno de instalación aquí nos dimos cuenta de que en modo rescate la partición root se
monta en el directorio target, ya que con esta terminal si podiamos usar el
comando ls nos dimos cuenta de que hacía falta el binario correspondiente a ls
en el directorio /bin por lo que copiamos el binario directamente de la ISO. \\

Una vez con pudimos usar el comando ls las cosas fueron más faciles, lo
primero que hicimos fue revisar los archivos importantes para el booteo
fstab, grub.cfg, etc. Al listar el contenido del directorio /etc
notamos que el archivo fstab tenía otro nombre por lo que lo corregimos,
ademas al abrirlo notamos que el contenido de este archivo estaba
comentado, por lo que restauramos la lineas necesarias para el arranque
del sistema. \\

El siguiente archivo que revisamos fue /boot/grub/grub.cfg el cual al
no encontrar en el directorio necesario, decidimos volverlo a generar,
posteriormente lo encontramos en otra ubicación pero como no lo necesitabamos ya
lo eliminamos. \\

Los siguientes pasos los hicimos después de clase: \\

Una vez arreglados los archivos anteriores decidimos intentar reiniciar
el sistema para verificar si las cosas iban bien, al reiniciar notamos
que ya podiamos hacer login, sin embargo nos dimos cuenta de que el
teclado estaba configurado con una distribución diferente (de francia), por
lo que para agilizar las cosas nuevamente entramos en el modo rescate de la
ISO de Debian y configuramos el teclado, ademas al recordar que no
contabamos con la contraseña de root, la cambiamos para poder iniciar
sesión. \\

Reiniciamos el sistema nuevamente y al instante notamos que el prompt
no se mostraba de forma correcta por lo que revisamos el archivo de
configuración de bash /etc/bash/bashrc y encontramos que las lineas
donde se definia el prompt estaban comentadas, por lo que corregimos esto
y volvimos a cargar bash para ver los cambios. Ademas revisamos el archivo
/etc/profile donde había lineas modificadas que definen la variable PS1
(variable del prompt) por lo que tambien corregimos esto. \\

Comparando los archivos y directorio del directorio raiz con los de una
instalación correcta nos dimos cuenta de que faltaba el directorio /mnt
por lo que lo creamos (aunque este no es indispensable). \\

Al querer eliminar el archivo anterior de grub.cfg que no nos percatamos
estaba con otro nombre dentro de /boot/grub nos dimos cuenta de que los
permisos para la mayoría de estos archivos había sido cambiados, por lo
que los restauramos comparandolos con los de la instalación correcta. \\

Antes de poder cambiar estos permisos nos dimos cuenta de que no podiamos
usar el comando chmod, hacia faltabel binario, por lo que nuevamente lo
copiamos desde la ISO montada. \\

En este punto todo parecía funcionar bien, por curiosidad y sobre todo
para buscar más errores en el sistema intentamos hacer ping a una
dirección ip lo cual fallo, por lo que revisamos los archivos de
configuración para las interfaces de red /etc/network/interfaces
el cual no existía y volvimos a crear. Reiniciamos el servicio de red
y solicitamos una dirección ip al servidor dhcp e intentamos nuevamente
hacer ping, está vez con éxito. \\

Otra vez todo parecía funcionar bien, por lo que pasamos bastante rato
buscando nuevos errores, al no encontrarlos decidimos iniciar sesión con
el usuario user, por lo que establecimos una nueva contraseña para este. \\

Al intentar iniciar sesión esta se cerraba automaticamente, por lo que
la lógica nos llevo a revisar el archivo /etc/passwd buscando errores,
ahí encontramos que en la linea correspondiente a user en lugar de tener
la ruta a un shell tenía nologin por lo que corregimos esto, con lo que
pudimos iniciar sesión con dicho usuario. \\

Al iniciar sesión notamos que no funcionaba el comando ls ya que imprimía
:v al usarlo, ya que el de root funcionaba correctamente no se trataba
de un problema con el binario, por lo que revisamos el archivo
/home/user/.bashrc el cual tenía un alias definido para ls, por lo que
lo quitamos. \\

Finalmente buscamos más errores en apt o en algunos otros de los binarios
más usados pero no encontramos nada por lo que terminamos. \\





\newpage

\textbf{Bibliografía} \\

\begin{itemize}

\item \url{}
\end{itemize}


\end{document}
