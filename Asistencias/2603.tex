\documentclass[11pt, a4paper]{report}

\usepackage[utf8]{inputenc}
\usepackage{fancyvrb}
\usepackage{enumitem}
\usepackage{hyperref}
\usepackage{multirow}
\usepackage{graphicx}

\begin{document}
\title{26/Marzo/2019}
\author{
  Carmona Mendoza Mat\'in\\
  \texttt{313075977}
}
\date{}
\maketitle

Hoy en clase con el profesor platicamos acerca de algunos sistemas de archivo
en red estos sistemas permiten acceder a archivos remotos como si estuviesen en
un medio de almacenamiento local. Gracias al sistema de archivos de red un
ordenador cliente puede acceder a sistemas de archivos que exporta el servidor,
con independencia del sistema de archivos de disco que se utiliza en el
servidor. Dentro de estos sistemas de archivos platicamos de LDAP las cuales
son las siglas de Lightweight Directory Access Protocol. Este un protocolo a
nivel de aplicación que permite realizar consultas sobre un servicio de
directorio para poder buscar información. Un servicio de directorio es como una
base de datos en la que organizar y almacenar información con objetos de
distintas clases. Esta estructura organizada de forma jerárquica de los objetos
en el directorio, se consigue con la implementación de LDAP. \\

es un protocolo estándar que permite administrar directorios, esto es, acceder
a bases de información de usuarios de una red mediante protocolos TCP/IP. \\

Las bases de información generalmente están relacionadas con los usuarios,
pero, algunas veces, se utilizan con otros propósitos, como el de administrar
el hardware de una compañía. \\

El objetivo del protocolo LDAP, desarrollado en 1993 en la Universidad de
Michigan, fue reemplazar al protocolo DAP (utilizado para acceder a los
servicios de directorio X.500 por OSI) integrándolo al TCP/IP. Desde 1995, DAP
se convirtió en LDAP independiente, con lo cual se dejó de utilizar sólo para
acceder a los directorios tipo X500. LDAP es una versión más simple del
protocolo DAP, de allí deriva su nombre Protocolo compacto de acceso a
directorios. 

\newpage

\textbf{Bibliografía} \\

\begin{itemize}

\item \url{https://es.ccm.net/contents/269-protocolo-ldap}


\end{itemize}


\end{document}
