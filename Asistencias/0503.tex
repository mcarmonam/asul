\documentclass[11pt, a4paper]{report}

\usepackage[utf8]{inputenc}
\usepackage{fancyvrb}
\usepackage{enumitem}
\usepackage{hyperref}
\usepackage{multirow}

\begin{document}
\title{Clase 25: 05/Marzo/2019}
\author{
  Carmona Mendoza Mat\'in\\
  \texttt{313075977}
}
\date{}
\maketitle

En esta clase comenzamos a hablar sobre la instalación de Debian por red
desde una imagen pequeña.

Un CD de instalación por red o netinst es un único CD que posibilita que
instale el sistema completo. Este único CD contiene sólo la mínima
cantidad de software para comenzar la instalación y obtener el resto de
paquetes a través de Internet.

¿Qué es lo mejor para mí: el CD mínimo arrancable o los CDs completos?
Depende, pero pensamos que en muchos casos la imagen del CD mínimo es lo
mejor: sobre todo, sólo tiene que descargar los paquetes que seleccionó
para instalarlos en su máquina, lo que ahorra tanto tiempo como ancho de
banda. Por otro lado, los CDs completos son más recomendables cuando
tiene que instalar en más de una máquina, o en máquinas sin una conexión
libre a Internet.

¿Qué tipo de conexiones de red se pueden usar durante la instalación? La
instalación por red asume que usted cuenta con conexión a Internet. Puede
ser de varias maneras: con una conexión analógica PPP, ethernet, red
inalámbrica (con algunas restricciones), pero no por RDSI (¡lo sentimos!).

Lo primero en hacer fue desgargar la imagen pequeña de Debian en la
página oficial y montar la imagen de la siguiente manera. \\

mount -o loop debian-9.8.0-amd64-netinst.iso /mnt/ \\

Procediendo copiando los archivos initrd.gz y vmlinuz de la carpeta
install.amd a nuestra carpeta /boot/. \\

Después de eso reiniciamos la computadora entrando a GRUB.

\newpage

\textbf{Bibliografía} \\

\begin{itemize}

\item  \url{https://www.debian.org/CD/netinst/}
\end{itemize}


\end{document}
