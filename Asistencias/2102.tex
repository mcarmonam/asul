\documentclass[11pt, a4paper]{report}

\usepackage[utf8]{inputenc}
\usepackage{fancyvrb}
\usepackage{enumitem}

\begin{document}
\title{Clase 18: 21/Febrero/2019}
\author{
  Carmona Mendoza Mat\'in\\
  \texttt{313075977}
}
\date{}
\maketitle

En la clase de hoy seguimos viendo la construcción de paquetes, refiriendonos
a Debian. \\

hablamos del comando chroot que permite ejecutar un proceso bajo un directorio
raíz simulado, de manera que el proceso no puede acceder a archivos fuera de
ese directorio. \\

Mencionamos los archivos configure que simplemente son para realizar alguna
configuración a algun programa. Principalamente se encuentran en la carpeta
etc. \\

Vimos que apt-get build-dep siver para instalar todos los paquetes necesarios
para poder compilar un determinado programa o paquete.
\end{document}
