\documentclass[11pt, a4paper]{report}

\usepackage[utf8]{inputenc}
\usepackage{fancyvrb}
\usepackage{enumitem}

\begin{document}
\title{Clase 06: 05/Febrero/2019}
\author{
  Carmona Mendoza Mat\'in\\
  \texttt{313075977}
}
\date{}
\maketitle

El d\'ia de hoy vimos las p\'aginas de manual herramienta esencial en la administración de 
sistemas Linux, nos proporcionan de datos y ayuda cuando necesitamos informaci\'on detallada 
sobre el funcionamiento de alguna caracter\'istica o comando. \\

/usr/local/man: lugar en el que debes guardar las p\'aginas de manual para que no sean vistas.\\
/usr/local/share/man: lugar en el que se puede acceder a las p\'aginas de manual. \\

Tambi\'en vimos procesos, c\'omo matar procesos, c\'omo enciar señales etc.. 
  

\end{document}
