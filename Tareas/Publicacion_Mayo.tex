\documentclass[11pt, a4paper]{report}

\usepackage[utf8]{inputenc}
\usepackage{fancyvrb}
\usepackage{enumitem}
\usepackage{hyperref}
\usepackage{multirow}
\usepackage{graphicx}

\begin{document}
\title{Publicaciones del mes de Mayo}
\author{
  Carmona Mendoza Mat\'in\\
  \texttt{313075977}
}
\date{}
\maketitle

\section*{Publicaciones que seleccioné}
\begin{itemize}

\item Publicación 1.
  \begin{itemize}
  \item GNU Guix 1.0.0 Released.
  \item Linux Journal.
  \item 02 Mayo 2019.
  \item \url{https://bit.ly/31JnDSG}
  \end{itemize}
  
\item Publicación 2
  \begin{itemize}
  \item Las mejores herramientas para recuperar datos en linux.
  \item SoloLinux.
  \item 17 Mayo 2019. 
  \item \url{https://bit.ly/2XN2419}
  \end{itemize}

\item Publicación 3
  \begin{itemize}
  \item Windows 10 ejecutará un kernel Linux completo mediante WSL 2.
  \item muyLinux.
  \item 07 Mayo 2019.
  \item \url{https://bit.ly/31yOyQZ}
  \end{itemize}
\end{itemize}  

\section*{Descripción de la publicación}
Escogí la publicación numero 2, Las mejores herramientas para recuperar datos
en linux, ya que más de una vez he tenido algún problema con mi disco o sistema
de almacenamiento. \\

Pienso que cuando ocurre un error muy grave, lo primero que debemos hacer es
intentar sacar todos los datos importantes que guardábamos en el sistema. Y la
mayoria de las veces por falta de experiencia y de conocimientos damos por
perdidos nuestro documento o información que tenemor en el equipo. \\

La publicación lista algunas de las herramientas que nos ayudan para recuperar
información que ya dabamos por perdida. En la lista aparecen herramientas como:

\begin{itemize}
\item Testdisk
\item Trinity Rescue Kit
\item Foremost
\item ddRescue
\end{itemize}

\section*{Conceptos que tuve que investigar}
Gracias al curso asul aprendí bastantes cosas de Linux, por lo que pude entender
bastante bien el articulo y no tuve que buscar información adicional.

\section*{Personas a las que recomendaría el articulo}
Se lo recomiendo a todas las personas en general ya que es un problema con el
que por lo menos una vez hemos enfrentado aunque sí pienso que deberia tener
conocimiento sobre discos y formatos.

\end{document}
