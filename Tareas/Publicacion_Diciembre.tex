\documentclass[11pt, a4paper]{report}

\usepackage[utf8]{inputenc}
\usepackage{fancyvrb}
\usepackage{enumitem}
\usepackage{hyperref}
\usepackage{multirow}
\usepackage{graphicx}

\begin{document}
\title{Publicaciones del mes de Diciembre}
\author{
  Carmona Mendoza Mat\'in\\
  \texttt{313075977}
}
\date{}
\maketitle

\section*{Publicaciones que seleccioné}
\begin{itemize}

\item Publicación 1.
  \begin{itemize}
  \item Epic Games recently announced it's working on a free cross-platform
    service for 2019:
  \item Linux Journal.
  \item 14 Diciembre 2018.
  \item \url{https://bit.ly/2wWncWK}
  \end{itemize}
  
\item Publicación 2
  \begin{itemize}
  \item Comandos básicos en Linux / Unix.
  \item SoloLinux.
  \item 11 Diciembre 2018. 
  \item \url{https://bit.ly/2WLqCeh}
  \end{itemize}

\item Publicación 3
  \begin{itemize}
  \item Linux, ¿el sistema favorito de los desarrolladores?
  \item muyLinux.
  \item 03 Diciembre 2018.
  \item \url{https://bit.ly/31xSx05}
  \end{itemize}
\end{itemize}  


\section*{Descripción de la publicación}
La publicación que seleccioné fue la número 2, comandos básicos en Linux / Unix,
me interesó mucho porque su nombre es "comandos básicos" por lo que intuí que
sabría todo lo que se mencionaría en el articulo y me equivoqué, la mayoria de
los comandos ya los había utilizado por lo menos una vez desde que uso Linux
pero hubo varios que no. Gracias al articulo aumenté mi lista de comandos. \\

El artículo explica comandos agrupado por familias. Las familias y comandos son
los siguientes:

\begin{itemize}
\item Usuarios
  \begin{itemize}
  \item id
  \item passwd
  \item who
  \end{itemize}
\item Archivos
  \begin{itemize}
  \item ls
  \item cp 
  \item rm 
  \item mv
  \end{itemize}
\item Archivos de texto
  \begin{itemize}
  \item cat
  \item more 
  \item less 
  \item head
  \item tail
  \end{itemize}
\item Gestionar directorios
  \begin{itemize} 
  \item cd 
  \item pwd 
  \item ln 
  \item mkdir
  \item rmdir
  \end{itemize}
\item Transferencia de archivos
  \begin{itemize}
  \item ftp
  \item sftp
  \item scp
  \end{itemize}  
\end{itemize}

\section*{Conceptos que tuve que investigar}
No tuve que investigar ningun concepto ya que el articulo explica muy bien cada
comando, ademas de que ya tenia conocimient de los temas que se abordan. 

\section*{Personas a las que recomendaría el articulo}
Recomiendo este articulo para novatos y personas que recien se han cambiado a
Linux ya que son comandos que te serviran muchísimo.

\end{document}
