\documentclass[11pt, a4paper]{report}

\usepackage[utf8]{inputenc}
\usepackage{fancyvrb}
\usepackage{enumitem}
\usepackage{hyperref}
\usepackage{multirow}
\usepackage{graphicx}

\begin{document}
\title{28/Febrero/2019}
\author{
  Carmona Mendoza Mat\'in\\
  \texttt{313075977}
}
\date{}
\maketitle

En la clase de hoy vimos con el ayudante la instalación de Arch. Arch es una
distribución Linux para computadoras x86 64 orientada a usuarios avanzados. Se
compone en su mayor parte de software libre y de codigo abierto (FOSS)​ y
apoya la participación comunitaria. \\

El enfoque de diseño del equipo de desarrollo, sigue el principio KISS como
línea general, y se centra en la pobreza , exactitud, minimalismo y
simplicidad, y espera que el usuario esté dispuesto a realizar un esfuerzo por
entender el funcionamiento del sistema. El gestor de paquetes escrito
específicamente para Arch, llamado Pacman, se usa para instalar, eliminar y
actualizar paquetes. \\

Arch Linux utiliza un modelo de actualización continua, de tal manera que una
actualización regular del sistema operativo es todo lo que se necesita para
obtener la última versión del software; las imágenes de instalación son
simplemente instantáneas de los principales componentes del sistema. \\

Arch Linux define simplicidad como ...una estructura base compacta sin añadidos
innecesarios, modificaciones, o complicaciones, que permite a un usuario
individual modificar el sistema de acuerdo a sus propias necesidades. La
simplicidad de su estructura no implica sencillez en su manejo.\\

\textbf{Instalación} \\

El sitio web de Arch Linux proporciona Imágenes ISO arrancables, que se pueden
ejecutar desde CD o USB en las arquitecturas soportadas. Un simple script de
líneas de comando (pacstrap) es usado para instalar el sistemas base. \\

A diferencia de otras distribuciones, la instalación del sistema base de Arch
Linux no provee de un Entorno de escritorio ni un Servidor gráfico instalados y
funcionales al usuario, no obstante este lo puede instalar desde los
repositorios. Tampoco provee de un cargador de arranque, que debe ser
configurado adecuadamente para que el sistema funcione, y pueda convivir con
otros sistemas operativos (si los hubiera) en el equipo informático. Los
paquetes adicionales pueden ser instalados con pacstrap o pacman después
iniciar el nuevo sistema. \\

Una alternativa al uso de imágenes de CD o USB para la instalación es utilizar
la versión estática del gestor de paquetes Pacman, desde otro sistema operativo
basado en Linux, mediante una técnica llamada Bootstrapping.​ El usuario
puede montar su partición formateando la unidad, e invocando a Pacman mediante
la línea de comandos puede utilizar el punto de montaje del dispositivo como
root para sus operaciones. De esta manera, el grupo de paquetes base y los
paquetes adicionales se pueden instalar en la partición recién formateada.


\newpage

\textbf{Bibliografía} \\

\begin{itemize}

\item \url{https://es.wikipedia.org/wiki/Arch_Linux#Instalación}


\end{itemize}


\end{document}
