\documentclass[11pt, a4paper]{report}

\usepackage[utf8]{inputenc}
\usepackage{fancyvrb}
\usepackage{enumitem}
\usepackage{hyperref}
\usepackage{multirow}
\usepackage{graphicx}

\begin{document}
\title{20/Marzo/2019}
\author{
  Carmona Mendoza Mat\'in\\
  \texttt{313075977}
}
\date{}
\maketitle

En la clase con el profesor hablamos sobre la herramienta axel. Axel es un
comando que sirve como acelerador de descargas. ¿Cómo lo hace? Abre más de una
conexión HTTP o FTP por cada intento de descarga, de tal modo que cada una de
ellas transfiere su propia parte del archivo. El comando es particularmente
útil cuando los servidores limitan el ancho de banda disponible por conexión. \\

Una vez más, la distribución de referencia será Ubuntu. \\

sudo apt-get install axel \\

En otros sabores de Linux puedes descargar y compilar el código fuente de axel,
si no es que ya está incluido en los repositorios correspondientes. \\

\section*{Virtualización y Emulación}

También hablamos sobre virtualización y emulación. \\

Un emulador es un programa de software que simula la funcionalidad de otro
programa o un componente de hardware. Dado que implementa funcionalidad por
software, proporciona una gran flexibilidad y la capacidad de recopilar
información muy detallada acerca de la ejecución. \\

El programa incluso podría ser escrito para una arquitectura diferente a
aquella sobre la cual el emulador se ejecuta –como ser, ejecutar un programa de
Android escrito para ARM en un emulador sobre un host x86. El inconveniente de
la emulación es que la capa de software incurre en una penalización en el
rendimiento, que debe ser cuidadosamente administrada a fin de crear un sistema
escalable. \\

Con la virtualización, el programa huésped se ejecuta realmente en el hardware
subyacente. El software de virtualización (VMM, Virtual Machine Monitor) sólo
media en los accesos de las diferentes máquinas virtuales al hardware real.
Así, éstas son independientes, y pueden ejecutar programas a velocidad casi
nativa. \\

Sin embargo, un programa en ejecución ocupa los recursos físicos reales, y como
resultado, el VMM y el sistema de análisis no pueden ejecutarse
simultáneamente, volviendo un desafío a la recolección detallada de datos.
Además, es difícil lograr ocultar por completo el VMM de las rutinas de
detección embebidas en las muestras de malware. \\

\newpage

\textbf{Bibliografía} \\

\begin{itemize}

\item \url{https://hipertextual.com/archivo/2010/06/comando-linux-axel-un-ligero-acelerador-de-descargar-http/}

\item \url{https://www.welivesecurity.com/la-es/2014/07/28/virtualizacion-o-emulacion-esa-es-la-cuestion/}

\end{itemize}


\end{document}
