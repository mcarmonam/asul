\documentclass[11pt, a4paper]{report}

\usepackage[utf8]{inputenc}
\usepackage{fancyvrb}
\usepackage{enumitem}
\usepackage{hyperref}
\usepackage{multirow}
\usepackage{graphicx}

\begin{document}
\title{Examen}
\author{
  Carmona Mendoza Martín\\
  \texttt{313075977}
}
\date{}
\maketitle

Este examen consiste en instalar una versión con más de 7 años de antiguedad de
alguna distribución de Linux y ver cuáles son las diferencias entre esta y la
versión más actual de la distribución en cuestión. \\

Yo comencé con la distribución FreeBSD. Busqué qué versión fue la primera y las
versiones son las siguientes: \\

\begin{figure}[!ht] 
    \begin{center}  
      \includegraphics[width=0.8\textwidth]{1.png} 
      \caption{} 
    \end{center} 
  \end{figure} 

Al ver que la primera distribución fue la 4.5 decidí buscar el iso de esa
versión en la web, lo encontré pero al parecer ya no estaba funcionando la
página. \\

  \begin{figure}[!ht] 
    \begin{center}  
      \includegraphics[width=0.8\textwidth]{2.png} 
      \caption{Error de página} 
    \end{center} 
  \end{figure}

  Y así ocurría con las siguientes versiones, excepto con las más recientes pero
  esas no las necesitaba así que decidí buscar otra distribución. \\

  Me tardé algo de tiempo en encontrar una distribución con la cuál no pasara lo
  mismo que la anterior ya que probé con Tiny Core y OpenMandriva y sucedía lo
  mismo. \\

  Después de una larga busqueda encontré Lubuntu 11.10 Oneiric Ocelot que fue
  una versión lanzada el 13 de octubre de 2011 y fue la primera versión como
  miembro oficial de la familia Ubuntu.  \\

  El archivo de 690 MB se descargo en 6 minutos y 37 segundos con una conexión
  de 94.36 Mbs. \\

  Entonces comencé con la instalación de Lubuntu 11.10 en una máquina virtual
  (VirtualBox). \\

  En mi máquina anfitrión tengo instalado Ubuntu Budgie que fue lanzado en
  Octubre del año 2018 (muy reciente), y noté que la instalación es identica, a
  pesar de que hay exactamente 7 años de diferencia el proceso no cambia, aquí
  dejo unas capturas de la instalación. \\

  \begin{figure}[!ht] 
    \begin{center}  
      \includegraphics[width=0.8\textwidth]{3.png} 
      \caption{} 
    \end{center} 
  \end{figure}

  \begin{figure}[!ht] 
    \begin{center}  
      \includegraphics[width=0.8\textwidth]{4.png} 
      \caption{} 
    \end{center} 
  \end{figure}

  \begin{figure}[!ht] 
    \begin{center}  
      \includegraphics[width=0.8\textwidth]{5.png} 
      \caption{} 
    \end{center} 
  \end{figure}

  \begin{figure}[!ht] 
    \begin{center}  
      \includegraphics[width=0.8\textwidth]{6.png} 
      \caption{} 
    \end{center} 
  \end{figure}

  \begin{figure}[!ht] 
    \begin{center}  
      \includegraphics[width=0.8\textwidth]{7.png} 
      \caption{} 
    \end{center} 
  \end{figure}

  \begin{figure}[!ht] 
    \begin{center}  
      \includegraphics[width=0.8\textwidth]{8.png} 
      \caption{} 
    \end{center} 
  \end{figure}

  \begin{figure}[!ht] 
    \begin{center}  
      \includegraphics[width=0.8\textwidth]{9.png} 
      \caption{} 
    \end{center} 
  \end{figure}

  \begin{figure}[!ht] 
    \begin{center}  
      \includegraphics[width=0.8\textwidth]{10.png} 
      \caption{} 
    \end{center} 
  \end{figure}

  \begin{figure}[!ht] 
    \begin{center}  
      \includegraphics[width=0.8\textwidth]{11.png} 
      \caption{} 
    \end{center}

    Una vez instalado me dediqué a ver las diferencias entre esta versión de
    Lubuntu con alguna versión mucho más reciente y encontré las siguientes: \\
    
  \end{figure}

    \begin{figure}[!ht] 
    \begin{center}  
      \includegraphics[width=1.0\textwidth]{13.png} 
      \caption{Versión del Kernel de Ubuntu 18.10} 
    \end{center} 
  \end{figure}

  \begin{figure}[!ht] 
    \begin{center}  
      \includegraphics[width=0.8\textwidth]{14.png} 
      \caption{Versión del Kernel Lubuntu 11.10} 
    \end{center}
  \end{figure}

      \begin{figure}[!ht] 
    \begin{center}  
      \includegraphics[width=0.8\textwidth]{16.png} 
      \caption{Versión de python en Ubuntu 18.10} 
    \end{center} 
  \end{figure}

  \begin{figure}[!ht] 
    \begin{center}  
      \includegraphics[width=0.8\textwidth]{15.png} 
      \caption{Versión de Lubuntu 11.10} 
    \end{center}
     \end{figure}

      \begin{figure}[!ht] 
    \begin{center}  
      \includegraphics[width=0.8\textwidth]{17.png} 
      \caption{Versión de gcc en Ubuntu 18.10} 
    \end{center} 
  \end{figure}

  \begin{figure}[!ht] 
    \begin{center}  
      \includegraphics[width=0.8\textwidth]{18.png} 
      \caption{Versión de gcc en Lubuntu 11.10} 
    \end{center}

    Después de eso noté que es no cambia mucho con versiones más actuales ya que
    cuenta con aplicaciones similares o iguales. Por ejemplo cuenta con
    aplicaciones de usuario como: \\

    \begin{itemize}
    \item Abiword, procesador de texto.
    \item Audacious, reproductor de música.
    \item Evince, lector de PDF.
    \item File-roller, archivador.
    \item Firefox, navegador web.
    \item Galculator, calculadora.
    \item GDebi, instalador de paquetes.
    \item GNOME Software, tienda de programas.
    \item Gnumeric, hoja de cálculo.
    \item Guvcview, cámara web.
    \item Synaptic y Centro de software de Lubuntu, gestores de paquetes.
    \item Actualización de software.
    \end{itemize}

    Aplicaciones para Internet: \\

    \begin{itemize}
    \item Firefox, navegador web.
    \item Quassel, cliente de IRC.
    \item Chromium
    \end{itemize}
    
  \end{figure}




\end{document}
