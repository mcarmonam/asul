\documentclass[11pt, a4paper]{report}

\usepackage[utf8]{inputenc}
\usepackage{fancyvrb}
\usepackage{enumitem}
\usepackage{hyperref}
\usepackage{multirow}
\usepackage{graphicx}

\begin{document}
\title{Clase 39: 25/Marzo/2019}
\author{
  Carmona Mendoza Mat\'in\\
  \texttt{313075977}
}
\date{}
\maketitle

Hoy fue el día de la Jornada de Orientación vocacional, entonces chicos del
grado de preparatoria asistieron a la clase para ver qué era lo que
realizabamos. \\

Prácticamente el profesor les platicó a los estudiantes qué era lo que se veía
en la materia. También les habló de cómo es que estabamos trabajando (nuestro
grupo). \\

Al finalizar la clase se habló de las similitudes y diferencias entre Linux y
Microsoft, esto porque todos los estudiantes que acudieron a la clase usan
Linux. \\

\section*{Algunas Diferencias}

Linux \\

\begin{itemize}
\item Es un sistema operativo abierto.
\item Lo desarrolla una comunidad mundial en individuos que lo hace gratuito.
\item La información siempre es accesible.
\item Logro de grandes avances en todo el mundo por su buena información.
\item Linux tiene un pequeño problema que un poco dificultoso al entrar en la
  red.
\item Contiene virus, pero muy pocos en comparación con Microsoft.
\item Se distribuye por GPL (General Public License).
\item Se enfoca mas a los derechos del usuario.
\item Permite la copia y modificación de información.
\end{itemize}

Microsoft \\

\begin{itemize}
\item Es un código cerrado.
\item Más fácil de configurar y crear.
\item Esta red contiene virus muy peligrosos como el (TROYANO) y el (MALWARE).
\item La licencia de Microsoft (EULA) limita al usuario.
\item Protegido por la empresa (EULA).
\item No es gratuito puesto que es privado.
\end{itemize}

\section*{Mitos sobre Linux}

\begin{itemize}
\item ¿Si uso Linux me quedaré aislado del resto?: Se ha hecho un gran esfuerzo
  en integrar a Linux a los ambientes corporativos multiplataformas y los
  resultados son muy satisfactorios. Los sistemas con Linux pueden integrase a
  un dominio Windows y hacer uso de los recursos compartidos. Se pueden editar
  archivos de Word y Excel en Linux e imprimirlos en la impresora del XP y
  viceversa. También se puede revisar Hotmail y usar el Messenger o ICQ en un
  equipo Linux. De igual manera los PDF y todos los tipos de archivos de
  imagen, video y audio están soportados; todo sin problemas.
\item Linux no está estandarizado: De todos los mitos, quizá éste es el más
  infundado. Linux es la plataforma que más busca sujetarse a los estándares.
  Existen estándares para todo, desde aquellos que definen como se debe
  comportar un manejador de ventanas hasta el formato de las hojas de cálculo,
  y los desarrolladores de Linux son muy respetuosos de apegarse a todas estas
  reglas.
\item En el software libre no hay innovación: La mejor innovación que han hecho
  los sistemas abiertos es el mismo Internet: el protocolo TCP/IP, que le da
  vida a la red, fue desarrollado por el equipo BSD de Berkeley y fue liberado
  bajo la BSD License , mientras el deficiente protocolo NetBeui ha sido
  abandonado. También fue en el softw are libre donde se dió la primer CLI
  (Common Lenguaje Interface) que fue Jython, años antes que el tardío .NET de
  Microsoft. Los Weblogs también son una innovación libre. El respaldo
  distribuido de información y sistemas de monitoreo de redes están también
  entre las muchas innovaciones libres.
\item Todo mundo puede ver el código de los programas libres y por eso son
  inseguros: En realidad, pasa todo lo contrario. Existen dos tipos de esquemas
  de seguridad: la tipo plaza pública en la cual todo mundo puede ver los
  detalles de un programa y cuando se encuentra una falla, avisa a todo mundo;
  y la tipo torre de marfil, donde sólo un reducido grupo puede ver el programa
  y cuando encuentra una falla no avisa a nadie. Al ser revisados por muchas
  personas y hacer públicas las fallas, es difícil que un error grave no sea
  detectado en un programa de software libre. En los programas torre de marfil,
  en cambio, las fallas pasan desapercibidas por el pequeño grupo y cuando las
  encuentran, no avisan. En todo caso, y quizá lo más grave, la respuesta de
  los programas tipo torre de marfil es muy lenta, arreglar un grave fallo de
  se guridad puede tardar m eses sin que los usuarios estén conscientes del
  peligro que corren, como ya ha pasado en varias ocasiones con Windows 2000 y
  XP. En la plaza pública, al ser dada a conocer una vulnerabilidad, uno puede
  decidir continuar con ese programa o reemplazarlo por otro que cumpla la
  misma función. En la torre de marfil, no se tiene esa libertad.
\item Sólo un experto programador puede instalar y usar Linux: Otro mito
  infundado. Cualquier persona puede ser un usuario eficiente de Linux. Si su
  empresa compra un equipo con Linux pre instalado, usted encenderá el equipo,
  usará el quemador, leerá sus e-mails, imprimirá sus documentos, escuchara
  música, navegara por Internet y al final de día apagará la computadora e irá
  a casa (Linux es famoso por su gran estabilidad). Todo ello sin saber una
  jota de programación.
\item Linux es feo: Linux ha cambiado mucho, sobre todo en los tres últimos
  años. Hasta la versión 7.2 de Mandrake , que salió al público amediados del
  2000, Linux o más precisamente, sus escritorios principales, KDE y Gnome ,
  adolecieron de un desarrollo gráfico que fuera al mismo ritmo que su
  desarrollo técnico. No todo era culpa de los escritorios, un aspecto
  fundamental del ambiente gráfico, el desplegado de las fuentes, lo realiza el
  XServer, y su implementación no era la mejor. \\

  Todo eso cambió al inicio del 2001 con el nuevo KDE y las mejoras al XServer.
  Note que estoy hablando del 2001, de esto hace cuatro años. Actualmente Linux
  posee uno de los entornos gráficos más atractivos de la industria (más
  atractivo que el de Windows y sólo superado por el MacOSX de Apple ). El
  ambiente gráfico de Linux es también, y por mucho, el más flexible y
  personalizable. Con frecuencia diseño páginas usando CSS y es una sorpresa
  para mí ver que en Windows las fuentes pierden definición, mientras en Linux
  se ve nítidas. \\

  Pero todo tiene un precio, Linux se dio a conocer en 1994 como el SO más
  rápido del mundo, lo que era cierto. Esa rapidez, lógicamente, se ha perdido
  con los escritorios corporativos. No obstante , aún existen los hacker's
  desktops como Window maker, Fluxbox o FVW M, que son entornos funcionales y
  visualmente atractivos.

\end{itemize}


\newpage

\textbf{Bibliografía} \\

\begin{itemize}

\item \url{https://elrincndemanu.wordpress.com/2015/10/06/semejanzas-y-diferencias-entre-apple-microsoft-y-linux/}
\item \url{http://www.somoslibres.org/modules.php?name=News&file=article&sid=541&fbclid=IwAR3vuUQGIWLPla9BCxdP1aG-szEapfJcUQA3vARyTbAnucrMqdrT0O9ya50}
\end{itemize}


\end{document}
