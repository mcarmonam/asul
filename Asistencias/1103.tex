\documentclass[11pt, a4paper]{report}

\usepackage[utf8]{inputenc}
\usepackage{fancyvrb}
\usepackage{enumitem}
\usepackage{hyperref}
\usepackage{multirow}
\usepackage{graphicx}

\begin{document}
\title{11/Marzo/2019}
\author{
  Carmona Mendoza Mat\'in\\
  \texttt{313075977}
}
\date{}
\maketitle

En la clase de hoy con el profesor vimos qué era el protocolo DHPC siendo una
extensión del protocolo Bootstrap (BOOTP) desarrollado en 1985 para conectar
dispositivos como terminales y estaciones de trabajo sin disco duro con un
Bootserver, del cual reciben su sistema operativo. El DHCP se desarrolló como
solución para redes de gran envergadura y ordenadores portátiles y por ello
complementa a BOOTP, entre otras cosas, por su capacidad para asignar
automáticamente direcciones de red reutilizables y por la existencia de
posibilidades de configuración adicionales. \\

La asignación de direcciones con DHCP se basa en un modelo cliente-servidor: el
terminal que quiere conectarse solicita la configuración IP a un servidor DHCP
que, por su parte, recurre a una base de datos que contiene los parámetros de
red asignables. Este servidor, componente de cualquier router ADSL moderno,
puede asignar los siguientes parámetros al cliente con ayuda de la información
de su base de datos:

\begin{itemize}
\item Dirección IP única.
\item Máscara de subred.
\item Puerta de enlace estándar.
\item Servidores DNS.
\item Configuración proxy por WPAD (Web Proxy Auto-Discovery Protocol).
\end{itemize}

El Dynamic Host Configuration Protocol tiene un punto débil y es su capacidad
para ser manipulado fácilmente. Como el cliente hace un llamamiento a
discreción a todos los servidores DHCP que podrían responder a su petición, a
un atacante le sería relativamente sencillo entrar en la red y hacerse pasar
por uno de ellos si tuviera acceso a ella. Este denominado servidor DHCP
“Rogue” (corrupto) intenta adelantarse con su respuesta al servidor legítimo y
si tiene éxito envía parámetros manipulados o inservibles. Si no envía puerta
de enlace, asigna una subred a cada cliente o responde a todas las peticiones
con la misma dirección IP, este atacante podría iniciar en la red un ataque de
denegación de servicio o Denial of Service.

\newpage

\textbf{Bibliografía} \\

\begin{itemize}

\item \url{https://www.ionos.mx/digitalguide/servidores/configuracion/que-es-el-dhcp-y-como-funciona/}

\end{itemize}


\end{document}
