\documentclass[11pt, a4paper]{report}

\usepackage[utf8]{inputenc}
\usepackage{fancyvrb}
\usepackage{enumitem}
\usepackage{hyperref}
\usepackage{multirow}
\usepackage{graphicx}

\begin{document}
\title{Clase 47: 05/Abril/2019}
\author{
  Carmona Mendoza Mat\'in\\
  \texttt{313075977}
}
\date{}
\maketitle

Hoy instalamos centOs 7 en una maquina virtual. \\

CentOS es un proyecto de código abierto gratuito de nivel empresarial con la
misma funcionalidad, rendimiento y estabilidad que el sistema operativo de pago
Redhat Enterprise Linux (RHEL). \\

CentOS comparte casi el 95\% de las características de la RHEL comercial con
la gran diferencia de la falta de puerto IBM System z y algunas variantes
limitadas para la virtualización. El otro diferenciador para la mayoría de los
administradores de TI es que el sitio de Centos.org carece de lo pulido de Red
Hat y no tiene tantas aplicaciones soportadas, proyectos terminados o
habilidades de gestión. Dicho esto, hay muchos beneficios sobresalientes para
CentOS 7 que se discuten a continuación. \\

\begin{enumerate}
\item CentOS 7 es compatible con la estrategia de redistribución del proveedor
  y obtiene soporte completo de la industria con actualizaciones de seguridad y
  material de capacitación. De hecho, CentOS es el único sistema operativo
  compatible con el popular panel de control de alojamiento web cPanel.

\item Cuando CentOS 7 está configurado correctamente y se ejecuta en hardware
  de calidad, es un sistema operativo de servidor muy estable, con muy pocos
  (si es que hay) problemas. Se reduce el riesgo de caídas y errores, ya que
  sólo ejecuta versiones estables de software empaquetado.

\item Con la distribución CentOS Linux puede obtener la ventaja del software de
  servidor de código abierto como Apache Web Server, Samba, Sendmail, CUPS,
  vsFTPd, MySQL y BIND.

\item Puede mejorar el rendimiento y el equilibrio de carga de los recursos
  configurando los equipos para que funcionen de forma colectiva, con un grupo
  de servidores que comparten un sistema de archivos común y que ofrecen
  aplicaciones de alta disponibilidad.

\item Los usuarios de CentOS 7 tienen acceso a características de seguridad
  actualizadas a nivel de empresa, incluyendo un potente firewall y el
  mecanismo de políticas SELinux.

\item Con una nueva instalación de CentOS, los usuarios obtienen soporte a
  largo plazo durante seis años, con actualizaciones de seguridad y parches
  críticos mantenidos durante una década después del lanzamiento inicial.

\item La plataforma CentOS 7 goza de una estabilidad superior a largo plazo con
  menos errores y agujeros de seguridad en comparación con otras distribuciones
  del mercado, por lo que no necesita nuevas versiones o actualizaciones de
  hardware con tanta frecuencia.

  Estas son solo algunas de las características que hacen que CentOS siga
  siendo una de las distribuciones de Linux más populares para servidores web,
  con velocidad, estabilidad y rendimiento mejorado sobre sus pares. Cuando se
  trata de sistemas operativos a nivel de empresa, no podría pedir nada mejor
  en el mundo Open Source.
\end{enumerate}

\newpage

\textbf{Bibliografía} \\

\begin{itemize}

\item \url{https://www.internetya.co/servidores-linux-ventajas-del-sistema-operativo-centos-7/}
\end{itemize}


\end{document}
