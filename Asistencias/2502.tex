\documentclass[11pt, a4paper]{report}

\usepackage[utf8]{inputenc}
\usepackage{fancyvrb}
\usepackage{enumitem}
\usepackage{hyperref}

\begin{document}
\title{Clase 20: 25/Febrero/2019}
\author{
  Carmona Mendoza Mat\'in\\
  \texttt{313075977}
}
\date{}
\maketitle

En la clase de hoy comenzamos a hablar de todos los dispositivos que usan un
kernel de linux. En realidad yo me imaginaba que Linux tenia un gran impacto
en la sociedad pero con esta platica observe que sí. \\

Investigué y estos son algunos dispositivos que utilizan Linux:

\begin{itemize}
\item  2013 Cadillac XTS : este coche emplea un reconocimiento de voz, muchos
  coches similares aceptan también comandos o direcciones pero este coche es
  lo suficientemente inteligente para “entender” la orden completa de una
  dirección al momento. Otra característica que tiene es la integración de
  búsquedas locales, si te apetece tomar un chocolate caliente, basta con
  decirlo en voz alta y el sistema te llevará a un lugar donde lo sirvan y
  todo este sistema esta hecho en Linux. 
\item Gran colisionador de hadrones : Es el proyecto físico más importante en
  la historia del universo conocido. Empezó sobre el 2008 y tomó como 20 años
  construirlo. Este invento de 10 Billones de Dolares funciona con Linux.
\item Control de tráfico de alta tecnología de San Francisco : El año pasado
  desarrrollaron un ordenador que cumplía los Estándares de control de
  transportes (ATC). Linux es usado como el sistema que encaja este sistema,
  pues se necesita un sistema robusto, expansible y que sea compatible con
  software de múltiples vendedores.
\item Calculadora Nspire CAS CX : Es un simple y poderoso dispositivo con
  arquitectura ARM basada en gráficos funcionando con Linux
\item Routers inalambricos : Un incontable número de routers wifi de compañias
  conocidas como Linksys, Asus, Netgear, etc funcionan con Linux.
\item SmartTV: muchos vemos en los anuncios estos inventos en la televisón, e
  incluso muchas funcionan con Ubuntu dentro
\end{itemize}

Despues hablamos de la carpeta etc/default la cual contiene algunos de los
parámetros que el usuario o administrador es probable que el cambio, más que
la incorporación de los valores en el real de los scripts de arranque. De esta
manera, los cambios persisten incluso si usted actualice el paquete y el script
de inicio se sustituye. \\

En debian, /etc/default/ es un directorio de la mayoría de archivos vacíos. La
forma en que está destinado a trabajar es que cada /etc/init.d/test script
primer fuentes /etc/default/test antes de iniciar/detener el servicio de
prueba. La finalidad del fichero es la de proporcionar más opciones para
iniciar el servicio, o anular determinados aspectos del servicio de inicio.




\newpage

\textbf{Bibliografía} \\

\begin{itemize}

\item  \url{http://www.nexolinux.com/15-dispositivos-que-funcionan-con-linux/}

\item \url{https://www.enmimaquinafunciona.com/pregunta/48817/cual-es-el-proposito-de-etcdefault}

\end{itemize}


\end{document}
