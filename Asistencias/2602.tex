\documentclass[11pt, a4paper]{report}

\usepackage[utf8]{inputenc}
\usepackage{fancyvrb}
\usepackage{enumitem}
\usepackage{hyperref}

\begin{document}
\title{Clase 21: 26/Febrero/2019}
\author{
  Carmona Mendoza Mat\'in\\
  \texttt{313075977}
}
\date{}
\maketitle

En la clase de hoy con el profesor continuamos hablando de directorios hablamos
de la carpeta /etc/default/grub que contiene la siguiente información:

\begin{itemize}
\item GRUB\_DEFAULT = 0 : Con la opción 0 hacemos que se seleccione por defecto
  la primera entrada (sistema), con la opción 1, la segunda… así sucesivamente.
  Si en cambio ponemos saved, hacemos que siempre se seleccione el último
  sistema al que se accedió.
\item \#GRUB\_HIDDEN\_TIMEOUT=0 : Si descomentamos esta línea (quitamos la
  almohadilla o hash, \#), oculta el menú de entradas del GRUB. Si ponemos un
  tiempo mas alto lo que hace es esconder el menú, pero esperarse un tiempo
  hasta continuar. Si en cambio comentamos la línea (añadimos un \#), mostrará
  el menú de entradas del GRUB.
\item GRUB\_HIDDEN\_MENU\_QUIET = true/false : Si está a “true” oculta la
  cuenta atrás , mientras que si está a “false” muestra la cuenta atrás
  (aparecerá en la zona inferior de la pantalla).
\item GRUB\_TIMEOUT = 90 : Esta línea indica el tiempo de espera (en segundos)
  hasta iniciar el sistema que tenemos como DEFAULT. Con un valor de -1, se
  desactiva la cuenta atrás y el valor será infinito.
\item GRUB\_CMDLINE\_LINUX\_DEFAULT = ”quiet splash acpi\_osi = Linux” : En esta
  línea, la opción quiet agrupa las entradas iguales, con splash, tras elegir
  el sistema, se muestra la imagen de carga en vez de los mensajes del kernel
  (la típica pantalla negra con letras blancas donde se muestra el proceso de
  encendido y comprobación de las particiones de los discos duros). La tercera
  opción, acpi\_osi=Linux puede arreglar varios problemas de hardware en nuestro
  sistema Linux, como por ejemplo el brillo, tal y como explico en este otro
  post.
\item \#GRUB\_GFXMODE = 640x480 : Esta línea permite cambiar la resolución del
  GRUB. Si se descomenta la línea, se aplica la resolución que pongamos en la
  línea. Para saber las resoluciones soportadas por nuestro GRUB, debemos poner
  vbeinfo en la línea de comandos del GRUB (pulsar c para acceder a ella). Eso
  sí, solo se pueden utilizar resoluciones permitidas por nuestra tarjeta
  gráfica. Lo más normal es poner el valor “auto” tras el igual.
\item \#GRUB\_DISABLE\_LINUX\_RECOVERY = ”true” : Al descomentar esta línea, no
  aparezcerá la opción de recovery mode de los sistemas Linux en el menú.
\end{itemize}

Después hablamos del archivo profile.d que en pocas palabras es un documento
que contiene la configuración del entorno (rutas, variables de entorno,
aliases, etc) \\

Hablamos también de algo que me pareció muy importante, algo llamado keylogger,
siendo un tipo de software o un dispositivo hardware específico que se encarga
de registrar las pulsaciones que se realizan en el teclado, para posteriormente
memorizarlas en un fichero o enviarlas a través de internet. \\

Hay dos tipos de keyloggers:

\begin{itemize}
\item Keylogger con Hardware
  \begin{itemize}
  \item Adaptadores : en línea que se intercalan en la conexión del
    teclado, tienen la ventaja de poder ser instalados inmediatamente.
    Sin embargo, mientras que pueden ser eventualmente inadvertidos se
    detectan fácilmente con una revisión visual detallada
  \item Dispositivos : que se pueden instalar dentro de los teclados
    estándares, requiere de habilidad para soldar y de tener acceso al
    teclado que se modificará. No son detectables a menos que se abra el
    cuerpo del teclado.
  \end{itemize}
\item Keylogger con Software
  \begin{itemize}
  \item Basado en núcleo: este método es el más difícil de escribir, y
    también de combatir. Tales keyloggers residen en el nivel del núcleo
    y son así prácticamente invisibles. Derriban el núcleo del sistema
    operativo y tienen casi siempre el acceso autorizado al hardware que
    los hace de gran alcance. Un keylogger que usa este método puede
    actuar como driver del teclado por ejemplo, y accede así a cualquier
    información registrada en el teclado mientras que va al sistema
    operativo.
  \item Enganchados: estos keyloggers registran las pulsaciones de las
    teclas del teclado con las funciones proporcionadas por el sistema
    operativo. El sistema operativo activa el keylogger en cualquier
    momento en que se presione una tecla, y realiza el registro.
  \item Métodos creativos: aquí el programador utiliza funciones como
    GetAsyncKeyState, GetForegroundWindow, etc. Estos son los más fáciles
    de escribir, pero como requieren la revisión del estado de cada tecla
    varias veces por segundo, pueden causar un aumento sensible en uso de
    la CPU y pueden ocasionalmente dejar escapar algunas pulsaciones del
    teclado.

  \end{itemize}
  
\end{itemize}

\newpage

\textbf{Bibliografía} \\

\begin{itemize}

\item  \url{http://linuxenandalu.com/2013/01/22/editar-el-menu-de-arranque-grub2/}

\item \url{https://es.wikipedia.org/wiki/Keylogger}

\end{itemize}


\end{document}
