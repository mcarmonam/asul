\documentclass[11pt, a4paper]{report}

\usepackage[utf8]{inputenc}
\usepackage{fancyvrb}
\usepackage{enumitem}

\begin{document}
\title{Clase 12: 13/Febrero/2019}
\author{
  Carmona Mendoza Mat\'in\\
  \texttt{313075977}
}
\date{}
\maketitle

La clase de hoy empezamos hablando de EFI, mecionando que MAC cuenta con este sistema por los procesadores Intel. \\

Vimos que los archivos con terminación .cao le dice al kernel cómo comunicarse con los dispositivos. \\

Mencionamos los log el cual te proporciona información de programas o dispositivos. \\

Linux es configurable puedes agragar tipos de letras, cambiar temas, personalizar el inicio, etc. \\

Comando df -lh nos muestra todo lo que está montado localmente. \\

Debemos ser cuidadosos con los diferentes tipos de archivos  del sistema porque podriamos sobreescribir algun
archivo dañando el sistema. \\

Vimos que también al software malicioso para linux entrando al sistema mediante una vulnerabilidad en archivos. \\

Telnet conexioón en modo texto.

\end{document}
