\documentclass[11pt, a4paper]{report}

\usepackage[utf8]{inputenc}
\usepackage{fancyvrb}
\usepackage{enumitem}

\begin{document}
\title{Clase 13: 14/Febrero/2019}
\author{
  Carmona Mendoza Mat\'in\\
  \texttt{313075977}
}
\date{}
\maketitle

En la clase de hoy comenzamos viendo el manejo de procesos con los comandos
bg (background) y fb (foreground). \\

bg: lanza el proceso pausado en segundo plano (similar a ejecutarlo
con \& al final, dejando el terminal libre). \\

fg: lanza el proceso pausado en primer plano (monopolizando el terminal). \\

Observamos algunas diferencias entre BIOS y EUFI. Algunas de ellas,
seguridad campatibilidad, capacidad, velocidad, etc. \\

Vimos también la definición de linter siendo una una herramienta que nos
ayuda a seguir las buenas prácticas o guías de estilo de nuestro código
fuente.\\

Vimos la sintaxis de las estructuras de control, IF, While y FOR. \\

passwd: Lugar donde se encuentran los usuarios. \\

shadow: lugar donde se encuentran las contraseñas.\\


\end{document}
