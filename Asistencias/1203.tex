\documentclass[11pt, a4paper]{report}

\usepackage[utf8]{inputenc}
\usepackage{fancyvrb}
\usepackage{enumitem}
\usepackage{hyperref}
\usepackage{multirow}
\usepackage{graphicx}

\begin{document}
\title{Clase 31: 12/Marzo/2019}
\author{
  Carmona Mendoza Mat\'in\\
  \texttt{313075977}
}
\date{}
\maketitle

Hoy en clase con el profesor continuamos trabajando con las máquinas
virtuales. Lo primero que hicimos fue configurar el adaptador de la máquina para
que fuera de tipo ”bridge” esto con el fin de que la máquina virtual fuera
visible desde cualquier otro equipo del segmento de la red. Para probar esto
usamos el comando ping para intentar comunicarnos con las otras máquinas
virtuales.
Una vez que verificamos las direcciones IP de las otras máquinas los siguiente
que hicimos fue conectarnos por medio de ssh con el usuario user ya que todas
las máquinas tienen este usuario con la misma contraseña, después de iniciar
sesión dejamos en el home del usuario un archivo que tenía por nombre nuestro
número de cuenta y contenía nuestro nombre y apellidos.
Posteriormente generamos las llaves publicas y privadas para la autenticación
en ssh con el comando ssh keygen, este comando genera dos archivos uno con
la llave publica y el otro con la llave privada, nuevamente usando ssh
agregamos la llave publica al archivo \/.ssh\/known hosts esto con el fin de
poder
realizar la conexión por medio de ssh sin tener que estar autenticándonos con
la contraseña cada vez que iniciemos sesión. \\

SSH o Secure Shell, es un protocolo de administración remota que permite a los
usuarios controlar y modificar sus servidores remotos a través de Internet. El
servicio se creó como un reemplazo seguro para el Telnet sin cifrar y utiliza
técnicas criptográficas para garantizar que todas las comunicaciones hacia y
desde el servidor remoto sucedan de manera encriptada. Proporciona un mecanismo
para autenticar un usuario remoto, transferir entradas desde el cliente al host
y retransmitir la salida de vuelta al cliente.



\newpage

\textbf{Bibliografía} \\

\begin{itemize}

\item \url{https://www.hostinger.mx/tutoriales/que-es-ssh}
\end{itemize}


\end{document}
