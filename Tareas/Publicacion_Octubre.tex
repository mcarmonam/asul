\documentclass[11pt, a4paper]{report}

\usepackage[utf8]{inputenc}
\usepackage{fancyvrb}
\usepackage{enumitem}
\usepackage{hyperref}
\usepackage{multirow}
\usepackage{graphicx}

\begin{document}
\title{Publicaciones del mes de Octubre}
\author{
  Carmona Mendoza Mat\'in\\
  \texttt{313075977}
}
\date{}
\maketitle

\section*{Publicaciones que seleccioné}
\begin{itemize}

\item Publicación 1.
  \begin{itemize}
  \item ProtonDB Reports 2671 Games Now Work on Linux.
  \item Linux Journal.
  \item 30 Octubre 2018.
  \item \url{https://bit.ly/2Ih6sjt}
  \end{itemize}
  
\item Publicación 2
  \begin{itemize}
  \item Debian, Ubuntu, and Other Distros are Leaving Users Vulnerable.
  \item Linux Magazine.
  \item 01 Octubre 2018. 
  \item \url{https://bit.ly/2ZwEztL}
  \end{itemize}

\item Publicación 3
  \begin{itemize}
  \item Los mejores grupos de Telegram sobre GNU/Linux y software libre en
    español.
  \item muyLinux.
  \item 29 Octubre 2018.
  \item \url{https://bit.ly/2FaeLMi}
  \end{itemize}
\end{itemize}  


\section*{Descripción de la publicación}
Elegí esta nota porque yo soy una persona amante de los videojuegos,
principalmente de las consolas, y debido a que la mayoria de los juegos estaban
en Windows tenía que tener instalados este sistema operativo en mi computador.

La noticia es muy corta pero habla que Valve Software lanzó Proton el cual está
integrado con Steam Play lo que facilita tener los juegos de Windows en Linux.
También comprende otras herramientas populares como Wine y DXVK, entre otras,
que un jugador tendría que instalar y mantener por sí mismo. Esto facilita
enormemente la carga para los usuarios de cambiar a Linux sin tener que
aprender los sistemas subyacentes o perder el acceso a una gran parte de Su
biblioteca de juegos. 

\section*{Conceptos que tuve que investigar}
En esta publicación entendí todo, no se me dificultó nada, los pocos terminos
computacionales que habian los conocia perfectamente. Ademas la publicación es
muy pequeña.


\section*{Personas a las que recomendaría el articulo}

Como mencioné anteriormente la publicación no cuenta con muchos tecnisismos por
lo que cualquier persona entendería a la perfección dicha nota pero lo
recomenría especialmente para las personas a las que les gusta mucho el mundo
del videojuego.


\end{document}
